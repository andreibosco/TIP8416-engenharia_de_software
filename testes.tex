% ==========================================================
% CAPITULO
% ==========================================================
\chapter{Testes}

% ----------------------------------------------------------
% Seção
% ----------------------------------------------------------
\section{Definição}

\begin{citacao}
Por melhores que sejam as técnicas de modelagem e especificação de software, por mais disciplinada e experiente que seja a equipe de desenvolvimento, sempre haverá um fator que faz que o teste de software seja necessário: o erro humano. É um mito pensar que bons desenvolvedores, bem concentrados e com boas ferramentas serão capazes de desenvolver software sem erros \apud{beizer1990}{WAZLAWICK2013}. % origem: página 287 do pdf
\end{citacao}


% % ----------------------------------------------------------
% % Seção
% % ----------------------------------------------------------
% \section{Verificação e Validação}

% Definição lorem ipsum dolor sit amet, consectetur adipisicing elit, sed do eiusmod
% tempor incididunt ut labore et dolore magna aliqua. Ut enim ad minim veniam,
% quis nostrud exercitation ullamco laboris nisi ut aliquip ex ea commodo
% consequat. Duis aute irure dolor in reprehenderit in voluptate velit esse
% cillum dolore eu fugiat nulla pariatur. Excepteur sint occaecat cupidatat non
% proident, sunt in culpa qui officia deserunt mollit anim id est laborum.


% \subsection{Análise estática}

% Definição lorem ipsum dolor sit amet, consectetur adipisicing elit, sed do eiusmod
% tempor incididunt ut labore et dolore magna aliqua. Ut enim ad minim veniam,
% quis nostrud exercitation ullamco laboris nisi ut aliquip ex ea commodo
% consequat. Duis aute irure dolor in reprehenderit in voluptate velit esse
% cillum dolore eu fugiat nulla pariatur. Excepteur sint occaecat cupidatat non
% proident, sunt in culpa qui officia deserunt mollit anim id est laborum.


% \subsection{Inspeções}


% \subsection{Revisões}


% \subsection{Caso de teste}

% ----------------------------------------------------------
% Seção
% ----------------------------------------------------------
\section{Tipos de Problemas}

% FIXME: página 32 de Software Product Quality Control

\begin{itemize}
    \item Falha (\emph{Failure / problem})
    \item Defeito (\emph{Fault / defect / bug})
    \item Erro (\emph{Error / mistake})
\end{itemize}

\begin{figure}[h]
    \centering
    \graphicspath{ {./graphics/} }
    \includegraphics[scale=0.8]{defeito_falha_erro}
    \caption{Visão geral de tipos de problemas}
    \label{fig:tipos_problemas}
    \legend{Fonte: \citeonline{wagner2013}}
\end{figure}

% ----------------------------------------------------------
% Seção
% ----------------------------------------------------------
\section{Categorias}
* em desenvolvimento *

\subsection{Estrutural}

\subsection{Funcional}



% ----------------------------------------------------------
% Seção
% ----------------------------------------------------------
\section{Técnicas}
* em desenvolvimento *

\subsection{White-box}

\subsection{Black-box}

\subsection{Grey-box}

% ----------------------------------------------------------
% Seção
% ----------------------------------------------------------
\section{Níveis}
* em desenvolvimento *

\subsection{Unitário / componente}

\subsection{Integração / operação do sistema / funcional}

\subsection{Validação}

\subsection{Sistema}

% ----------------------------------------------------------
% Seção
% ----------------------------------------------------------
\section{Foco do teste}
* em desenvolvimento *

\subsection{Conformidade}

\subsection{Configuração}

\subsection{Recuperação}

\subsection{Regressão}

\subsection{Estresse}

\subsection{Desempenho}

\subsection{Usabilidade}

\subsection{Acessibilidade}

\subsection{Internacionalização}

% ----------------------------------------------------------
% Seção
% ----------------------------------------------------------
\section{Ferramentas}
* em desenvolvimento *

% ----------------------------------------------------------
% Seção
% ----------------------------------------------------------
\section{Fluxo de Teste em Diferentes Metodologias}
* em desenvolvimento *