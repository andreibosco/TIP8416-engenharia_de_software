%% abtex2-modelo-trabalho-academico.tex, v-1.7.1 laurocesar
%% Copyright 2012-2013 by abnTeX2 group at http://abntex2.googlecode.com/ 
%%
%% This work may be distributed and/or modified under the
%% conditions of the LaTeX Project Public License, either version 1.3
%% of this license or (at your option) any later version.
%% The latest version of this license is in
%%   http://www.latex-project.org/lppl.txt
%% and version 1.3 or later is part of all distributions of LaTeX
%% version 2005/12/01 or later.
%%
%% This work has the LPPL maintenance status `maintained'.
%% 
%% The Current Maintainer of this work is the abnTeX2 team, led
%% by Lauro César Araujo. Further information are available on 
%% http://abntex2.googlecode.com/
%%
%% This work consists of the files abntex2-modelo-trabalho-academico.tex,
%% abntex2-modelo-include-comandos and abntex2-modelo-references.bib
%%

% ------------------------------------------------------------------------
% ------------------------------------------------------------------------
% abnTeX2: Modelo de Trabalho Academico (tese de doutorado, dissertacao de
% mestrado e trabalhos monograficos em geral) em conformidade com 
% ABNT NBR 14724:2011: Informacao e documentacao - Trabalhos academicos -
% Apresentacao
% ------------------------------------------------------------------------
% ------------------------------------------------------------------------

\documentclass[
	% -- opções da classe memoir --
	12pt,				% tamanho da fonte
	openright,			% capítulos começam em pág ímpar (insere página vazia caso preciso)
	oneside,			% para impressão em verso e anverso. Oposto a oneside
	a4paper,			% tamanho do papel. 
	% -- opções da classe abntex2 --
	%chapter=TITLE,		% títulos de capítulos convertidos em letras maiúsculas
	%section=TITLE,		% títulos de seções convertidos em letras maiúsculas
	%subsection=TITLE,	% títulos de subseções convertidos em letras maiúsculas
	%subsubsection=TITLE,% títulos de subsubseções convertidos em letras maiúsculas
	% -- opções do pacote babel --
	english,			% idioma adicional para hifenização
	brazil,				% o último idioma é o principal do documento
	]{abntex2}


% ---
% PACOTES
% ---

% ---
% Pacotes fundamentais 
% ---
\usepackage{cmap}				% Mapear caracteres especiais no PDF
\usepackage{lmodern}			% Usa a fonte Latin Modern			
\usepackage[T1]{fontenc}		% Selecao de codigos de fonte.
\usepackage[utf8]{inputenc}		% Codificacao do documento (conversão automática dos acentos)
\usepackage{lastpage}			% Usado pela Ficha catalográfica
\usepackage{indentfirst}		% Indenta o primeiro parágrafo de cada seção.
\usepackage{color}				% Controle das cores
\usepackage{graphicx}			% Inclusão de gráficos
\usepackage{abntex2-ufc}
% ---
		
% ---
% Pacotes adicionais, usados apenas no âmbito do Modelo Canônico do abnteX2
% ---
\usepackage{lipsum}				% para geração de dummy text
\usepackage{booktabs}
\usepackage[table,xcdraw]{xcolor}
% ---

% ---
% Pacotes de citações
% ---
\usepackage[brazilian,hyperpageref]{backref}	 % Paginas com as citações na bibl
\usepackage[alf]{abntex2cite}	% Citações padrão ABNT

% --- 
% CONFIGURAÇÕES DE PACOTES
% --- 

% ---
% Configurações do pacote backref
% Usado sem a opção hyperpageref de backref
\renewcommand{\backrefpagesname}{Citado na(s) página(s):~}
% Texto padrão antes do número das páginas
\renewcommand{\backref}{}
% Define os textos da citação
 \renewcommand*{\backrefalt}[4]{
 	\ifcase #1 %
 		Nenhuma citação no texto.%
 	\or
 		Citado na página #2.%
 	\else
 		Citado #1 vezes nas páginas #2.%
 	\fi}%
% ---


% ---
% Informações de dados para CAPA e FOLHA DE ROSTO
% ---
\titulo{Modelos de qualidade de produto de software com perspectivas sobre métodos ágeis, inspeções e testes}
\autor{Andrei Bosco Bezerra Torres}
\local{Fortaleza - CE}
\data{Abril de 2015}
\orientador[Orientadores]{Professor: José Marques Soares}
\instituicao{%
  Universidade Federal do Ceará -- UFC
  \par
  Departamento de Engenharia de Teleinformática
  \par
  Programa de Pós-Graduação em Engenharia de Teleinformática}
\tipotrabalho{Atividade de mestrado}
% O preambulo deve conter o tipo do trabalho, o objetivo, 
% o nome da instituição e a área de concentração 
\preambulo{Trabalho da disciplina de Engenharia de Software apresentado como parte das atividades.}
% ---


% ---
% Configurações de aparência do PDF final

% alterando o aspecto da cor azul
\definecolor{blue}{RGB}{41,5,195}

% informações do PDF
\makeatletter
\hypersetup{
     	%pagebackref=true,
		pdftitle={\@title}, 
		pdfauthor={\@author},
    	pdfsubject={\imprimirpreambulo},
	    pdfcreator={LaTeX with abnTeX2},
		pdfkeywords={abnt}{latex}{abntex}{abntex2}{trabalho acadêmico}, 
		colorlinks=true,       		% false: boxed links; true: colored links
    	linkcolor=blue,          	% color of internal links
    	citecolor=blue,        		% color of links to bibliography
    	filecolor=magenta,      		% color of file links
		urlcolor=blue,
		bookmarksdepth=4
}
\makeatother
% --- 

% --- 
% Espaçamentos entre linhas e parágrafos 
% --- 

% O tamanho do parágrafo é dado por:
\setlength{\parindent}{1.3cm}

% Controle do espaçamento entre um parágrafo e outro:
% \setlength{\parskip}{0.2cm}  % tente também \onelineskip
\setlength{\onelineskip}{0.2cm}

% ---
% compila o indice
% ---
\makeindex
% ---

% ----
% Início do documento
% ----
\begin{document}

% Retira espaço extra obsoleto entre as frases.
\frenchspacing 

% ----------------------------------------------------------
% ELEMENTOS PRÉ-TEXTUAIS
% ----------------------------------------------------------
% \pretextual

% ---
% Capa
% ---
\imprimircapa
% ---

% ---
% Folha de rosto
% (o * indica que haverá a ficha bibliográfica)
% ---
%\imprimirfolhaderosto*
\imprimirfolhaderosto
% ---

% ---
% RESUMOS
% ---

% resumo em português
\begin{resumo}
 O foco deste trabalho é abordar sobre qualidade de software e todo o processo envolvido no desenvolvimento e entrega de um software de alto padrão de qualidade, abordando os modelos de qualidade de produto, gestão, métricas e inspeção. Também será abordada a aplicação de modelos de qualidade em projetos que adotam métodos ágeis e como isso pode impactar o resultado final do produto.

 \vspace{\onelineskip}
    
 \noindent
 \textbf{Palavras-chaves}: modelos de qualidade de produto, qualidade de software, inspeção, testes, validação, verificação, métodos ágeis.
\end{resumo}

% ---
% inserir lista de ilustrações
% ---
\pdfbookmark[0]{\listfigurename}{lof}
\listoffigures*
\cleardoublepage
% ---

% ---
% inserir lista de tabelas
% ---
\pdfbookmark[0]{\listtablename}{lot}
\listoftables*
\cleardoublepage
% ---

% ---
% inserir lista de abreviaturas e siglas
% ---
\begin{siglas}
    \item[ABNT] Associação Brasileira de Normas Técnicas
    \item[COQUAMO] \emph{Constructive Quality Model} - Modelo de Qualidade Construtivo
    \item[IEC] \emph{International Electrotechnical Commission} - Comissão Eletrotécnica Internacional
    \item[IEEE] \emph{Institute of Electrical and Electronics Engineers} - Instituto de Engenheiros Eletricistas e Eletrônicos
    \item[ISO] \emph{International Organization for Standardization} - Organização Internacional para Padronização
    \item[QA] \emph{Quality Assurance} - Garantia de Qualidade
    \item[QMOOD] \emph{Quality Model for Object-Oriented Design} - Modelo de Qualidade
para Projeto Orientados a Objeto
    \item[SQuaRE] \emph{Systems and software Quality Requirements and Evaluation} - Avaliação e Requisitos de Qualidade de Sistemas e Software
\end{siglas}
% ---

% ---
% inserir lista de símbolos
% ---
% \begin{simbolos}
%   \item[$ \Gamma $] Letra grega Gama
%   \item[$ \Lambda $] Lambda
%   \item[$ \zeta $] Letra grega minúscula zeta
%   \item[$ \in $] Pertence
% \end{simbolos}
% ---

% ---
% inserir o sumario
% ---
\pdfbookmark[0]{\contentsname}{toc}
\tableofcontents*
\cleardoublepage
% ---



% ----------------------------------------------------------
% ELEMENTOS TEXTUAIS
% ----------------------------------------------------------
\textual

% ==========================================================
% CAPITULO
% ==========================================================
\chapter[Introdução]{Introdução}

Este trabalho fará uma abordagem sobre qualidade de software, iniciando uma contextualização sobre seus diversos tipos de definição e conceitos básicos e, posteriormente, focando nos modelos de qualidade existentes, desde os mais relevantes historicamente até o padrão atual ISO/IEC. Como fechamento será apresentada uma análise sobre como os modelos de processo de software abordam (ou não) a qualidade, com foco nas metodologias ágeis.
%FIXME continuar
%Este capítulo fará uma abordagem sobre diversos tipos de modelos de qualidade existente, incluindo uma breve contextualização histórica de cada, e contará também com seções introdutórias sobre gestão da qualidade, medição e padrões/certificações. E, como fechamento, será apresentado como a qualidade de software pode ser inserido em projetos que implementam metodologias ágeis.

% ----------------------------------------------------------
% Seção
% ----------------------------------------------------------
\section{Definição}

Definir \emph{qualidade de software}, ou mesmo \emph{qualidade} não é tão simples como parece. O conceito de qualidade está sob discussão por filósofos há mais de 2 mil anos, desde Platão. Você sabe identificar quando um produto é de qualidade, mas normalmente não é fácil indicar o que lhe dá essa característica. No caso de qualidade para Engenharia de Software, vários autores trazem suas definições e existem vários modelos e organizações que almejam criar uma estrutura para a criação de softwares de qualidade.

\begin{description}
    \item[Definição \citeonline{IEEE1990} para Qualidade:] \hfill
        \begin{itemize}
            \item O grau em que um sistema, componente ou processo alcança os requerimentos especificados
            \item O grau em que um sistema, componente ou processo atinge as necessidades ou expectativas de um cliente ou usuário
        \end{itemize}
    \item[Definição \citeonline{IEEEvocabulary2010} para Qualidade:] \hfill
        \begin{itemize}
            \item O grau em que um sistema, componente ou processo alcança os requerimentos especificados
            \item A habilidade de um produto, serviço, sistema, componente ou processo em atender as necessidades, expectativas ou requisitos do cliente ou usuário
            \item A totalidade das características de uma entidade que afetam sua capacidade de satisfazer necessidades explícitas e implícitas
            \item A conformidade com as expectativas dos usuários, a conformidade com os requisitos do usuário, a satisfação do cliente, confiabilidade e nível de defeitos presentes
            \item O grau em que um conjunto de características inerentes satisfaz os requisitos
            \item O grau em que um sistema, componente ou processo atinge as necessidades ou expectativas de um cliente ou usuário
        \end{itemize}
    \item[Definição NBR ISO/IEC 9126 (\citeyear{nbr9126}) para Qualidade:]\footnote{Substituída pela nova família de Normas NBR ISO/IEC 25000}
        \begin{itemize}
            \item  Totalidade de características de uma entidade que lhe confere a capacidade de satisfazer as necessidades explícitas e implícitas.
        \end{itemize}
    \item[Definição \citeonline{IEEE1990} para Garantia de Qualidade (QA):] \hfill
        \begin{itemize}
            \item Um padrão planejado e sistemático de todas as ações necessárias para garantir uma confiança adequada de que um item ou produto está em conformidade com os requisitos técnicos estabelecidos.
            \item Um conjunto de atividades destinadas a avaliar o processo pelo qual os produtos são desenvolvidos ou produzidos.
        \end{itemize}
    \item[Definição de \citeonline{PRESSMAN2010} para Qualidade de Software:] \hfill
        \begin{itemize}
            \item Um processo de software eficaz aplicado de uma maneira que cria um produto útil, que oferece valor mensurável para quem produz e para quem usa.
        \end{itemize}
\end{description}

Note a evolução das definições da IEEE entre 1990 e 2010. Por que é tão difícil chegar a uma definição objetiva? Basicamente porque é quase impossível analisar objetivamente se um requisito foi cumprido adequadamente ou não. Pode haver divergências de interpretação de um requisito entre o desenvolvedor e o cliente ou pode haver conflito de expectativas entre diferentes \emph{stakeholders}\footnote{Qualquer pessoa ou organização que tenha interesse, ou seja afetado pelo projeto.}. Então em que se deve pensar para chegar à um produto com qualidade?

\section{Fatores de Qualidade}

Devido aos questionamentos levantados anteriormente, cabe à equipe de gerenciamento de qualidade julgar se um nível de qualidade foi alcançado. \citeonline{SOMMERVILLE2011} exemplifica algumas questões que auxiliam a validar a qualidade de um sistema:
\begin{enumerate}
    \item Os padrões de programação e documentação foram seguidos durante o processo de desenvolvimento?
    \item O software foi testado adequadamente?
    \item O software é suficientemente confiável para ser posto em produção?
    \item A performance do software é aceitável para uso normal?
    \item O software é utilizável?
    \item O software é bem estruturado e compreensível?
\end{enumerate}


% REF: página 673 de Software Engineering (9th Edition) - rever este parágrafo
\citeonline{SOMMERVILLE2011} lembra que a análise da qualidade de um software não está sujeita apenas a analisar se uma funcionalidade foi implementada corretamente, mas também depende das características não-funcionais. Se uma funcionalidade não funcionar como esperado, o usuário normalmente encontra outras maneiras de realizar o que deseja, porém, se o software for lento ou instável, ele será impedido de alcançar seus objetivos.

Ele também reforça que é impossível um sistema ser otimizado em todos estes pontos, portanto é necessário ser analisado e definido quais atributos são vitais para a aplicação. Caso a eficiência seja crítica, então as outras características podem ser sacrificadas, por exemplo.

% REF: ver página 430 de Software Engineering A Practitioner's Approach 7th Edition - Roger Pressman

Na visão de \citeonline{GARVIN1987}, a qualidade deve ser analisada de um ponto de vista multidimensional, desde a conformidade com os requisitos até o caráter estético da aplicação (ver \autoref{tab:qualidade_garvin}).

\begin{table}[h]
    \caption{Dimensões de qualidade de Garvin}
    \label{tab:qualidade_garvin}
    \begin{tabular}{p{3.7cm}|p{11cm}}
        \textbf{Performance} & o software entrega todo o conteúdo/funções/característica especificadas nos requerimentos e de maneira que agregue valor ao usuário? \\ \hline
        \textbf{Característica} & o software possui alguma característica que surpreenderá e agradará usuários finais novatos? \\ \hline
        \textbf{Confiabilidade} & o software disponibiliza suas funções sem falhas? \\ \hline
        \textbf{Conformidade} & o software está em conformidade com padrões locais e externos que são relevantes à aplicação? Ele segue convenções de código e de design? \\ \hline
        \textbf{Durabilidade} & O software pode ser mantido (alterado) e corrigido (debugado) sem efeitos colaterais indesejáveis? \\ \hline
        \textbf{Manutenibilidade} & o software pode ser mantido (alterado) e corrigido (debugado) em um espaço de tempo aceitável? \\ \hline
        \textbf{Estética} & o valor estético, elegância, a \emph{presença} do software.\\ \hline
        \textbf{Percepção} & Preconceitos (positivos ou negativos) que influenciam a percepção de qualidade. Um software anteriormente produzido por um desenvolvedor que produziu algo de baixa qualidade causará uma influência negativa na nova versão.\\
  \end{tabular}
  \legend{Fonte: \citeonline{PRESSMAN2010}}
\end{table}

Conforme dito anteriormente, a qualidade é um conceito extremamente amplo e abrangente, e nem sempre simples de ser avaliado objetivamente. Com isso, é necessário metas concretas, um modelo com o objetivo de descrever, avaliar e/ou prever qualidade \cite{wagner2013}. Diversos modelos serão descritos e analisados no capítulo \ref{cap:modelos_qualidade}.

% ----------------------------------------------------------
% Seção
% ----------------------------------------------------------
\section{Tipos de Qualidade}
% REF: página 21 do livro Software Product Quality Control

A qualidade de software é dividida em dois aspectos: a \emph{qualidade do produto} em si e a \emph{qualidade do processo}. A qualidade do produto foca no controle de qualidade para gerar um produto final que atenda aos requisitos e traga satisfação do usuário, enquanto a qualidade do processo foca no controle da produção para otimizar o desenvolvimento e o uso otimizado dos recursos disponíveis.

Os modelos de maturidade, que tem como foco a \emph{qualidade do processo}, não serão abordados, tendo este trabalho a \emph{qualidade do produto} e os modelos de qualidade criados para ele como foco principal.

\citeonline{WAZLAWICK2013} indica que apesar de não existir garantia que um bom processo vá resultar em um bom produto, em geral admite-se que a mesma equipe com um bom processo vá produzir produtos melhores do que se não tivesse processo algum \cite{WAZLAWICK2013}.

% fonte: http://www.bayt.com/en/specialties/q/52980/can-you-differentiate-between-product-quality-and-process-quality/
% Product quality is focusing on meeting tolerances in the end result of the manufacturing activities. The end result is measured on a standard of "good enough".
% Process quality focuses on each activity and forces the activities to achieve maximum tolerances irrespective of the end result. 
%For product quality we use quality control to assure product confirm to production standards and tolerance where for process quality we use quality assurance to assure process effectiveness and efficiency 

% REF: página 228 de Engenharia de Software - conceitos e práticas
% REF: berander2005, página 3
% REF: pressman2010, página 428, final da página
E dentro do contexto de qualidade do produto, \citeonline{PRESSMAN2010} define dois tipos de qualidade:

\begin{description}
    \item [Qualidade de projeto (atender as necessidades do cliente):] qualidade identificável através de características não mesuráveis objetivamente, a qualidade do produto em atender as expectativas do cliente, explícitas ou não.
    \item [Qualidade de conformação (conformidade com as especificações):] avalia quão \\ bem o produto atende aos requisitos, em conformidade com as especificações definidas, através de características mesuráveis.
\end{description}

% ----------------------------------------------------------
% Seção
% ----------------------------------------------------------
\section{Custo da qualidade}
\begin{citacao} % pagina 435 - pressman
Se você produzir um sistema de software de péssima qualidade, então você perderá pois ninguém irá comprá-lo. Por outro lado, se você gastar tempo infinito, extremo esforço e grandes quantidades de dinheiro para produzir um software absolutamente perfeito então você demorará tanto para concluí-lo e será tão dispendioso que você acabará fora do mercado do mesmo jeito. Ou você terá perdido a janela de oportunidade ou terá exaurido sua verba. Então as pessoas da indústria tentam encontrar o posicionamento mágico onde o produto é bom o suficiente para não ser rejeitado de cara, como durante a avaliação, e sem ser preso a um perfeccionismo e excesso de trabalho que faça com ele acabe demorando ou saindo caro demais. (MEYER, Bertrand apud \citeauthoronline{PRESSMAN2010})
\end{citacao}

% FIXME: continuar - pressman, 436

* em desenvolvimento *


% ==========================================================
% CAPITULO
% ==========================================================
\chapter{Modelos de Qualidade de Produto}
\label{cap:modelos_qualidade}

% ----------------------------------------------------------
% Seção
% ----------------------------------------------------------
\section{Definição}

Um \emph{Modelo de Qualidade} é um modelo cujo objetivo é descrever, avaliar e/ou prever qualidade. Muitos modelos de qualidade fazem uma decomposição da qualidade geral de um produto em sub-qualidades para que se torne mais fácil de compreender e gerenciar \cite{wagner2013}.

Este tópico vem sendo pesquisado à décadas e existe um grande número de modelos \cite{Klas2009} e seria inviável analisar todas, então este trabalho focará nos modelos mais relevantes, seja por sua importância histórica ou por ser um padrão internacional. \citeonline{wagner2013} propõe uma divisão entre os modelos de qualidade, que será utilizada como referência:
\begin{itemize}
    \item \emph{Hierárquicos}: Um modelo de decomposição hierárquica dos fatores de qualidade. Ex.: McCall, Boehm, ISO 9126;
    \item \emph{Meta-modelos}: Um modelo que define como modelos de qualidade válidos devem ser estruturados, cada um com propriedades e métricas específicas para o projeto. Ex.: Dromey, COQUAMO (\emph{Constructive Quality Model} - Modelo de Qualidade Construtivo), QMOOD (\emph{Quality Model for Object-Oriented Design} - Modelo de Qualidade para Projeto Orientados a Objeto);
    \item \emph{Estatístico ou implícito}: Um modelo estatístico que busca estimar ou prever os fatores de qualidade, através de ferramentas de análise ou de testes. Ex.: Modelo de confiabilidade de crescimento (\emph{reliability growth model}).
\end{itemize}

% ----------------------------------------------------------
% Seção
% ----------------------------------------------------------
\section{Modelos Hierárquicos}
\subsection{Modelo McCall (1977)}
% foco na qualidade do produto, dividido em visão externa (fatores de qualidade a especificar) e visão interna (critérios de qualidade para desenvolver)

Um dos predecessores mais renomado dos modelos de qualidade atuais é o modelo de \citeonline{general1977factors}, criado para o exército dos EUA e focado nos desenvolvedores de sistema e no processo de desenvolvimento. Este modelo define fatores de qualidade que tentam aproximar o ponto de vista do usuário e as prioridades dos desenvolvedores. Estes fatores são divididos em três perspectivas (ver \autoref{fig:mcCall-fatores_qualidade}):

\begin{itemize}
    \item \emph{Revisão do produto} (capacidade em sofrer alterações): inclui manutenibilidade (o esforço necessário para localizar e corrigir um erro), flexibilidade (facilidade de fazer alterações necessárias no produto) e testabilidade (a facilidade de testar o produto, garantindo que não possui erros e está de acordo com as especificações);
    \item \emph{Transição do produto} (adaptabilidade a novos ambientes): é focado na portabilidade (o esforço necessário para transferir o produto de um ambiente para outro), reusabilidade (a facilidade de reusar o produto em outro contexto) e interoperabilidade (o esforço de conectar o produto com outro);
    \item \emph{Operação do produto} (características operacionais): a qualidade operacional do produto depende da exatidão (o grau em que o produto segue as especificações), confiabilidade (a habilidade do produto em não falhar), eficiência (eficiência de execução, de armazenamento, do uso de recursos), integridade (proteção contra acessos não autorizados) e usabilidade (a facilidade de uso).
\end{itemize}

\begin{figure}[h]
    \centering
    \caption{Perspectivas e fatores de qualidade de McCall}
    \graphicspath{ {./graphics/} }
    \includegraphics[scale=0.8]{mcCall-fatores_qualidade}
    \label{fig:mcCall-fatores_qualidade}
    \legend{Fonte: \citeonline{PRESSMAN2010}}
\end{figure}

% REF: página 7 de software_quality_atributes
McCall define uma hierarquia (\autoref{fig:mcCall-quality_model}) relacionando os fatores citados acima com critérios:
\begin{itemize}
    \item 11 fatores (especificação): descrever a visão externa do software, do ponto de vista do usuário;
    \item 23 critérios de qualidade (construção): descrever a visão interna do software, do ponto de vista dos desenvolvedores;
    \item Métricas (controle): definidas e utilizados para prover método de medida.
\end{itemize}

Os fatores de qualidade retratam as características comportamentais do sistema, os critérios são atributos para um ou mais fatores, enquanto as métricas buscam capturar os aspectos do critério de qualidade \cite{berander2005}.

\cite{Naik2011} demonstra como definir as métricas:
\begin{itemize}
    \item Formular um conjunto de questões sobre o critério de qualidade que possam ser respondidas com \emph{sim} ou \emph{não};
    \item Dividir as respostas \emph{sim} pelo número total de questões para obter um valor entre 0 e 1. Esse valor será a métrica do atributo.
\end{itemize}

O ponto negativo desta proposta é a subjetividade de avaliação da métrica, que cabe ao julgamento da pessoa que respondeu as perguntas \cite{berander2005}, e alguns critérios podem levar a perguntas que não sejam facilmente respondidas apenas com 'sim' ou 'não' \cite{Naik2011}.

\begin{figure}[H]
    \centering
    \caption{Hierarquia de fatores e critérios de McCall}
    \graphicspath{ {./graphics/} }
    \includegraphics[scale=0.85]{mcCall-graph-andrei}
    \label{fig:mcCall-quality_model}
    \legend{Fonte: \citeonline{Naik2011} [Adaptação do autor]}
\end{figure}

% ----------------------------------------------------------
% Seção
% ----------------------------------------------------------
\subsection{Modelo Boehm (1978)}
% três níveis de características: alto nível (usuário), intermediária (software) e primitiva (métricas e avaliação)
% FIXME: indicar que ele criou o modelo em espiral?

Outro modelo histórico foi criado por \citeonline{BOEHM1978}, que, de maneira similar ao modelo de McCall, apresenta um modelo hierárquico de características. Para Boehm, o primeiro pode a ser analisado é a utilidade geral (\emph{General utility}) do software, pois se ele não for útil então todo o projeto é uma perda de tempo e dinheiro \cite{PFLEEGER2010}.

O modelo de Boehm é dividido em três níveis, onde cada uma contribui para a qualidade geral: \emph{alto nível} (usuário), \emph{intermediário} (software) e \emph{primitivas} (métricas). As características de alto nível tem como objetivo responder a três questões quanto ao uso do software pelo usuário, que não é necessariamente o cliente \cite{berander2005}:

% REF: página 614, pfleeger - 'Software Engineering Theory and Practice 4e Shari Pfleeger Joanne Atlee 9780136061694.pdf'
\begin{itemize}
    \item Utilidade 'como está' (\emph{As is utility}) - usuário final: quão fácil (eficiência, confiança, usabilidade) é de utilizá-lo tal como está? 
    \item Portabilidade - desenvolvedor: posso continuar a utilizá-lo caso altere o meu ambiente? Caso o compilador seja substituído, o programa não deve sofrer degradação de funcionalidades, por exemplo.
    \item Manutenibilidade - desenvolvedor mantenedor: quão fácil é entender, modificar e testar?
\end{itemize}

Já as características de nível médio representam os 7 fatores de qualidade esperados pelos desenvolvedores (ver \autoref{fig:boehm-hierarquia_modelo}), enquanto o nível de características primitivas provê as métricas de avaliação do sistema. Uma das principais diferenças deste modelo para o de McCall é que este abrange a performance do hardware como característica relevante à qualidade. Ou seja, para Boehm a qualidade do software deve satisfazer tanto as necessidades do usuário final quanto os desenvolvedores envolvidos, e como reflexo dessa qualidade o software deve: fazer o que o usuário requer; utilizar os recursos computacionais de maneira correta e eficaz; ser fácil de aprender e utilizar; e ser bem projetado, bem programado, facilmente testável e de fácil manutenção \cite{PFLEEGER2010}.

Porém, segundo \cite{SURYN2014}, apesar deste modelo ser um avanço comparado ao modelo de McCall, as características de utilidade geral (\emph{general utility}) e 'utilitário como está' (\emph{as-is utility}) ainda são conceitos genéricos e imprecisos, resultando em uma aplicabilidade limitada.

\begin{figure}[H]
    \centering
    \caption{Hierarquia do Modelo de Qualidade de Boehm}
    \graphicspath{ {./graphics/} }
    \includegraphics[scale=0.95]{boehm-hierarquia_modelo-andrei}
    \label{fig:boehm-hierarquia_modelo}
    \legend{Fonte: \citeonline{PFLEEGER2010} [Adaptação do autor]}
\end{figure}

% ----------------------------------------------------------
% Seção
% ----------------------------------------------------------
%\section{Modelo FURPS (\emph{Functionality, Usability, Reliability, Performance, Supportability})}
% FIXME: manter ou apagar?

% ----------------------------------------------------------
% Seção
% ----------------------------------------------------------
\subsection{Padrão ISO 9126 (1991)}
O padrão ISO 9126 (\emph{Software Product Evaluation: Quality Characteristics and Guidelines for their Use} - Avaliação do Produto de Software: Características de Qualidade e Diretrizes para sua Utilização) foi criado com base nos modelos de McCall e Boehm, focando em três aspectos da qualidade de software \cite{SURYN2014}:
\begin{description}
    \item [Qualidade no uso:] qualidade de uso é o ponto de vista do usuário quanto a qualidade do software ao ser utilizado em um ambiente e contexto específico. Ele avalia se o usuário consegue alcançar seus objetivos, e não as propriedades do software.
    \item [Qualidade externa:] características do software de um ponto de vista externo. A qualidade do software ao ser executado, medido e avaliado em um ambiente simulado ou com dados simulados, utilizando métricas externas.
    \item [Qualidade interna:] a totalidade dos atributos do software que determinam sua habilidade em satisfazer as necessidades implícitas e explícitas do usuário, em condições específicas.
\end{description}

Os fatores de qualidade interna e externas são divididos em uma hierarquia de 3 níveis, composto de características de qualidade, sub-características e métricas (ver \autoref{fig:iso9126-hierarquia_interno_externo} e \autoref{tab:iso9126-caracteristicas}). Mais de 200 métricas\footnote{Listagem de métricas do padrão estão disponíveis na ISO/IEC 9126-4} foram propostas como parte do padrão e outras métricas podem ser criadas e utilizadas de acordo com a necessidade \cite{SURYN2014}.

\begin{figure}[h]
    \centering
    \caption{Hierarquia de qualidade interna e externa do padrão ISO 9126}
    \graphicspath{ {./graphics/} }
    \includegraphics[scale=1.0]{iso9126-hierarquia_interno_externo-andrei}
    \label{fig:iso9126-hierarquia_interno_externo}
    \legend{Fonte: \citeonline{SURYN2014} [Adaptação do autor]}
\end{figure}

\begin{table}[h]
    \centering
    \caption{Características de qualidade ISO 9126}
    \label{tab:iso9126-caracteristicas}
    \begin{tabular}{p{3cm} p{12.0cm}}
        \toprule
        \textbf{Característica} & \textbf{Definição} \\ \midrule
        Funcionalidade & Um conjunto de atributos que evidenciam a existência de um conjunto de funções e suas propriedades especificadas. As funções são aqueles que satisfazem as necessidades declaradas ou implícitas \\
        \rowcolor[HTML]{EFEFEF}
        Confiabilidade & Um conjunto de atributos que afetam a capacidade de software de manter seu nível de desempenho sob condições estabelecidas por um período de tempo determinado. \\
        Usabilidade & Um conjunto de atributos que evidenciam o esforço necessário para o uso e à apreciação individual de tal uso por um conjunto de usuários explícito ou implícito. \\
        \rowcolor[HTML]{EFEFEF}
        Eficiência & Um conjunto de atributos que afetam a relação entre o desempenho do software e da quantidade de recursos utilizados sob condições estabelecidas. \\
        Manutibilidade & Um conjunto de atributos que evidenciam o esforço necessário para fazer modificações especificadas (que podem incluir correções, melhorias ou adaptações do software às mudanças no ambiente, mudanças nos requisitos e especificações funcionais). \\
        \rowcolor[HTML]{EFEFEF}
        Portabilidade & Um conjunto de atributos que afetam a capacidade do software de ser transferido de um ambiente para outro (incluindo ambiente organizacional, hardware ou software). \\ \bottomrule
    \end{tabular}
    \legend{Fonte: \citeonline{PFLEEGER2010} [Tradução do autor]}
\end{table}

Já os fatores de qualidade de uso são organizando em uma hierarquia de 2 níveis, composto das características de qualidade e as métricas (ver \autoref{fig:iso9126-hierarquia_qualidade_uso}). A junção e conexão dos três aspectos da qualidade de software do padrão ISO 9126 formam um modelo preditivo (ver \autoref{fig:iso9126-hierarquia_relacionamento}).

\begin{figure}[h]
    \centering
    \caption{Hierarquia de qualidade de uso do padrão ISO 9126}
    \graphicspath{ {./graphics/} }
    \includegraphics[scale=1.0]{iso9126-hierarquia_qualidade_uso-andrei}
    \label{fig:iso9126-hierarquia_qualidade_uso}
    \legend{Fonte: \citeonline{nbr9126} [Adaptação do autor]}
\end{figure}

Em comparação com os modelo de McCall e de Boehm, o padrão ISO 9126 apresenta as seguintes diferenças:
\begin{itemize}
    \item O padrão ISO 9126 e o modelo de Boehm abordam as características visíveis ao usuário, enquanto o modelo de McCall foca nas características internas;
    \item Um critério de qualidade no padrão ISO 9126 impacta apenas um fator, enquanto nos modelos de McCall e Boehm um critério pode impactar múltiplos fatores;
    \item Um fator de alto nível nos Modelos de McCall e de Boehm pode ser um fator de baixo nível no padrão ISO 9126 (ex.: \emph{testabilidade}, que é um fator no modelo de McCall e um critério dentro de \emph{manutibilidade} no padrão ISO 9126).
\end{itemize}

\begin{figure}[h]
    \centering
    \caption{Relacionamento entre aspectos de qualidade do padrão ISO 9126}
    \graphicspath{ {./graphics/} }
    \includegraphics[scale=1.0]{iso9126-hierarquia_relacionamento-andrei}
    \label{fig:iso9126-hierarquia_relacionamento}
    \legend{Fonte: \citeonline{nbr9126} [Adaptação do autor]}
\end{figure}

% ----------------------------------------------------------
% Seção
% ----------------------------------------------------------
\subsection{SQuaRE - ISO/IEC 25010 (\emph{Systems and software Quality Requirements
and Evaluation} - 2011)}
% FONTE: página 230 de Engenharia de Software - Conceitos e Práticas

Em 2011, a família de normas ISO/IEC 25000 substituíram a norma ISO 9126, sendo um dos fatores que a antiga norma aplicava-se apenas no processo de desenvolvimento e uso do produto de software, e não se preocupava quanto à definição do produto em si.

O novo modelo propõe duas perspectivas, cada uma com um modelo específico (ver \autoref{fig:iso25010-hierarquia_qualidade_completa}): a primeira perspectiva é relacionada ao uso do software, com um modelo chamado \emph{modelo de qualidade no uso}, e a segunda perspectiva é relacionada ao software em si, com um modelo chamado \emph{modelo de qualidade do produto de sistema/software}. O modelo de qualidade no uso é composto por 5 características e 11 métricas. O modelo de qualidade do produto é composto de 8 características, que são divididas em sub-características que se relacionam com as propriedades estáticas e dinâmicas do software.

Em comparação ao modelo de qualidades internas e externas do padrão ISO 9126, o modelo de qualidade do produto do padrão ISO 25010 utiliza terminologias diferentes mas que possuem significados similares \cite{SURYN2014}:

\begin{description}
    \item [Estática:] corresponde à qualidade interna da ISO 9126, que se aplica ao software em desenvolvimento, que não está sendo executado;
    \item [Dinâmico:] corresponde à qualidade externa da ISO 9126, que se aplica ao sistema e software em execução, mas não dentro do contexto operacional (para isto existe o modelo de qualidade no uso).
\end{description}

\citeonline{wagner2013} critica que o padrão não indica quando utilizar cada modelo, e que aparentemente as empresas adotam o modelo de qualidade em uso apenas quando analisam a usabilidade do software. Ele lembra também que esse modelo é uma taxonomia, e não é a única, e talvez nem a melhor, maneira de estruturar as características de qualidade do software, e que se o modelo for utilizado como algo além de um \emph{checklist} então não será o mais adequado.

\begin{figure}[H]
    \centering
    \caption{Hierarquias de qualidade do padrão ISO 25010}
    \graphicspath{ {./graphics/} }
    \includegraphics[scale=0.85]{iso25010-hierarquia_qualidade_completa-andrei}
    \label{fig:iso25010-hierarquia_qualidade_completa}
    \legend{Fonte: \citeonline{SURYN2014} [Adaptação do autor]}
\end{figure}

% \begin{figure}[h]
%     \centering
%     \caption{Hierarquia de qualidade de uso do padrão ISO 25010}
%     \graphicspath{ {./graphics/} }
%     \includegraphics[scale=0.9]{iso25010-hierarquia_qualidade_uso-andrei}
%     \label{fig:iso25010-hierarquia_qualidade_uso}
%     \legend{Fonte: \citeonline{SURYN2014} [Adaptação do autor] }
% \end{figure}

% % FIXME: fazer versão própria
% \begin{figure}[H]
%     \centering
%     \caption{Hierarquia de qualidade de produto do padrão ISO 25010}
%     \graphicspath{ {./graphics/} }
%     \includegraphics[scale=0.8]{iso25010-hierarquia_qualidade_produto}
%     \label{fig:iso25010-hierarquia_qualidade_produto}
%     \legend{Fonte: \citeonline{SURYN2014}}
% \end{figure}

% Para facilitar a compreensão das fases do modelo SquaRE veja a \autoref{tab:fases_square} que apresenta sua aplicação no ciclo de vida da normal ISO/IEC 15288 e nos modelos A, B e C. % FIXME: corrigir referência da tabela e inserir os outros modelos de projeto (RUP talvez?)

% \begin{table}[H]
%     \caption{Fases do modelo SQuaRE e sua aplicação no ciclo de vida de um projeto}
%     \label{tab:fases_square}
%     \begin{tabular}{p{4.9cm}|p{4.9cm}|p{4.8cm}}
%         \textbf{Fases da 15288} & \textbf{Fases de RUP} & \textbf{Fases de SQuaRE} \\ \hline
%         Definição de requisitos dos interessados & Blabla & Requisitos de qualidade do produto de software\\ \hline
%         Análise dos requisitos & Blabla & Desenvolvimento do produto \\
%         \emph{Design} arquitetural & Blabla & \\
%         Implementação & Blabla & \\
%         Integração & Blabla & \\
%         Verificação & Blabla & \\
%         Transição & Blabla & \\
%         Validação & Blabla & \\ \hline
%         Operação & Blabla & Uso do produto \\
%         Manutenção & Blabla & \\ \hline
%         Aposentadoria & Blabla & -
%   \end{tabular}
%   \legend{Fonte: \citeonline{WAZLAWICK2013}}
% \end{table}

% ----------------------------------------------------------
% Seção
% ----------------------------------------------------------
\subsection{Análise dos modelos hierárquicos}

Os modelos apresentados são úteis para auxiliar na análise da qualidade do software desenvolvido, porém nenhum dos modelos explica a lógica de porque algumas características foram consideradas e outras não, porque um atributo está em uma posição da hierarquia, porque nenhum dos modelos aborda segurança, dentre outras questões \cite{PFLEEGER2010}.

\citeonline{wagner2013} também critica a ambiguidade das características de qualidade e a dificuldade da realização de uma medição adequada. Uma pesquisa realizada por \citeonline{wagner2012} apontou que menos de 28\% das empresas pesquisadas implementam tais modelos hierárquicos, e que 71\% delas desenvolveram uma variante própria, o que demonstra uma necessidade de customização.

\citeonline{PRESSMAN2010} concorda com estas críticas, mas lembra que esses modelos proveem uma boa base para medições indiretas e como um checklist para avaliar a qualidade de um sistema.

\begin{table}[H]
\centering
    \caption{Comparação entre critérios/objetivos dos modelos de qualidade de McCall, Boehm e ISO 9126}
    \label{tab:modelos_comparacao}
    \begin{tabular}{@{}lllll@{}}
        \toprule
        \textbf{Critérios/Objetivos} & \multicolumn{1}{c}{McCall} & Bohem & ISO 9126 & ISO 25010\\ \midrule
        Exatidão & * & confiabilidade & funcionalidade & funcionalidade \\
        \rowcolor[HTML]{EFEFEF} 
        Confiabilidade & * & * & * & * \\
        Integridade & * &  &  & segurança \\
        \rowcolor[HTML]{EFEFEF} 
        Usabilidade & * & compreensibilidade & * & * \\
        Eficiência & * & * & * & * \\
        \rowcolor[HTML]{EFEFEF} 
        Manutibilidade & * & * & * & * \\
        Testabilidade & * &  & manutibilidade & manutibilidade \\
        \rowcolor[HTML]{EFEFEF} 
        Interoperabilidade & * &  &  & compatibilidade \\
        Flexibilidade & * &  &  & manutibilidade \\
        \rowcolor[HTML]{EFEFEF} 
        Reusabilidade & * &  &  & manutibilidade\\
        Portabilidade & * & * & * & * \\
        \rowcolor[HTML]{EFEFEF} 
        Clareza &  & * &  & \\
        Modificabilidade &  & * & manutibilidade & manutibilidade \\
        \rowcolor[HTML]{EFEFEF} 
        Documentação &  & * &  & \\
        Resiliência &  & * &  & confiabilidade \\
        \rowcolor[HTML]{EFEFEF} 
        %Compreensibilidade &  & * &  & usabilidade \\ % remover?
        Validez &  & * & funcionalidade & funcionalidade \\
        Funcionalidade &  &  & * & * \\
        \rowcolor[HTML]{EFEFEF} 
        Generalidade &  & * &  & \\
        Segurança & & & funcionalidade & * \\ \bottomrule
    \end{tabular}
    \legend{Fonte: \citeonline{berander2005} [Adaptação do autor]}
\end{table}

% ----------------------------------------------------------
% Seção
% ----------------------------------------------------------
\section{Meta-modelos}
\subsection{Modelo Dromey (1995)}
% três princípios fundamentais: atributos de qualidade, propriedades do produto, e as ligações entre eles

O modelo de \citeonline{Dromey1995} traz uma diferente abordagem de McCall e Boehm, pois propõe que o modelo de qualidade seja baseado na perspectiva do produto, na qual a avaliação e características de qualidade irão variar para cada produto \cite{SURYN2014}:

\begin{citacao}
O que deve ser compreendido em qualquer tentativa de criar um modelo de qualidade é que um software não manifesta diretamente os atributos de qualidade. No entanto, ele exibe características de produto que implicam ou contribuem para atributos de qualidade ou outras características (defeitos do produto) que detraem dos atributos de qualidade do produto. A maioria dos modelos de qualidade de software falham em lidar adequadamente com a faceta de problema nas características do produto, e eles também falham em relacionar diretamente os atributos de qualidade e as características correspondentes no produto. [Tradução do autor] \cite{Dromey1995}
\end{citacao}

Seguindo esta lógica, Dromey criou um \emph{framework} de avaliação de qualidade que analisa a qualidade de componentes através de propriedade tangíveis. Cada artefato produzido durante a criação do software (ex.: código, documentação, guia de uso, etc) pode ser associado a um modelo de qualidade diferente (uma variável pode ser um componente do modelo de implementação, e um módulo pode ser um componente do modelo de projeto, por exemplo) \cite{SURYN2014}. E as propriedades tangíveis são divididas em quatro tipo de propriedades (ver \autoref{fig:dromey-estrutura_modelo}):

\begin{itemize}
    \item exatidão: verifica se algum princípio básico foi violado
    \item interno: quão bem um componente foi implementado de acordo com o seu uso pretendido
    \item contextual: lida com influências externas e o uso do componente
    \item descritiva: quão bem descrito é o componente
\end{itemize}

Essas propriedades são utilizadas para a avaliar qualidade dos componentes, e para Dromey, um software de alta qualidade é o resultado do somatório de componentes de alta qualidade, desde os requerimentos individuais até a nomenclatura das variáveis do código. O modelo de exemplo ilustrado na \autoref{fig:dromey-estrutura_modelo_exemplo} foi criado seguindo os seguintes passos \cite{PFLEEGER2010}:

\begin{enumerate}
    \item identificar os atributos de qualidade de alto nível
    \item identificar os componentes do produto
    \item identificar e classificar as propriedades mais significantes e tangíveis para cada componente
    \item propor um conjunto de axiomas conectando as propriedades do produto com os atributos de qualidade
    \item avaliar o modelo, identificando suas fraquezas, refinando-o e recriando-o
\end{enumerate}

Contudo, \citeonline{SURYN2014} faz uma analogia interessante: utilizar farinha, maçã e canela de alta qualidade resultará uma torta de maçã de alta qualdiade? Óbvio que não, outros elementos são necessários: uma receita (arquitetura e processo de execução), as preferências do usuário (fator totalmente ignorado por Dromey) e pessoas com as qualificações e ferramentas para executar a receita apropriadamente. Certamente este modelo pode ser classificado como uma abordagem \emph{bottom-to-top} de qualidade de software.

\begin{figure}[H]
    \centering
    \caption{Estrutura do Modelo de Qualidade de Dromey}
    \graphicspath{ {./graphics/} }
    \includegraphics[scale=0.85]{dromey-estrutura_modelo-andrei}
    \label{fig:dromey-estrutura_modelo}
    \legend{Fonte: \citeonline{Deissenboeck2009} [Adaptação do autor]}
\end{figure}

\begin{figure}[H]
    \centering
    \caption{Avaliação de qualidade de um componente de variável}
    \graphicspath{ {./graphics/} }
    \includegraphics[scale=0.85]{dromey-estrutura_modelo_exemplo-andrei}
    \label{fig:dromey-estrutura_modelo_exemplo}
    \legend{Fonte: \citeonline{SURYN2014} [Adaptação do autor]}
\end{figure}

% ----------------------------------------------------------
% Seção
% ----------------------------------------------------------
\subsection{COQUAMO}

% ----------------------------------------------------------
% Seção
% ----------------------------------------------------------
\section{Modelos estatísticos}
\subsection{Modelo de confiabilidade de crescimento}

% ==========================================================
% CAPITULO
% ==========================================================
\chapter{Gestão da Qualidade}

% ----------------------------------------------------------
% Seção
% ----------------------------------------------------------
\section{Definição}
A gestão da qualidade e suas técnicas de controle de qualidade (QA) foram desenvolvidas para, em conjunto com novas tecnologias de desenvolvimento e testes, resolver o problema da baixa qualidade dos software que permeou o séc. XX \cite{SOMMERVILLE2011}. Ele pode ser efetuada pelo gerente do projeto, mas preferencialmente é realizada por um gerente e/ou equipe especializada para que possa ter uma visão objetiva e independente da equipe de desenvolvimento \cite{SOMMERVILLE2011}. % FIXME: como isso funcionaria em uma metodologia ágil, com pequenas equipes?

\citeonline{WAZLAWICK2013} apresenta um modelo de maturidade organizacional em relação à qualidade, definido por Crosby (1979), baseado em cinco estágios:
\begin{enumerate}
    \item \emph{Desconhecimento:} quando a empresa sequer sabe que tem problemas com qualidade. Não há compreensão de que a qualidade seja um objetivo, ferramentas não são usadas ou conhecidas, e inspeções de qualidade não são realizadas.
    \item \emph{Despertar:} a empresa reconhece que tem problemas com a qualidade e que precisa começar a lidar com eles, mas ainda vê isso como um mal necessário, não como fonte de lucro.
    \item \emph{Alinhamento:} o gerenciamento de qualidade se torna uma ferramente institucional e os problemas vão sendo priorizados e resolvidos à medida que surgem.
    \item \emph{Sabedoria:} a prevenção de problemas, e não apenas sua correção, torna-se rotina na empresa. Problemas são identificados antes que surjam, e todos os processos e rotinas estão abertos a mudanças visando à melhoria da qualidade.
    \item \emph{Certeza:} a gestão da qualidade é uma constante e parte essencial do funcionamento da empresa. Quase todos os problemas são prevenidos e eliminados antes de surgirem.
\end{enumerate}

Para realizar a verificação da qualidade são utilizadas técnicas como \emph{revisões, inspeções} e \emph{testes sistemáticos}. \citeonline{SOMMERVILLE2011} reforça que o objetivo das revisões e inspeções não é avaliar a performance da equipe de desenvolvimento, mas detectar erros, que inevitavelmente ocorrerão. A equipe responsável de QA deve ser sensível às questões individuais dos desenvolvedores, e deve criar uma cultura de suporte à descoberta de erros sem implicação de culpa.

\subsection{\emph{Walkthrough}}


\subsection{Inspeções Fagan}


\subsection{Método \emph{Cleanroom}}

% ----------------------------------------------------------
% Seção
% ----------------------------------------------------------
\section{Medição da Qualidade}

% ----------------------------------------------------------
% Seção
% ----------------------------------------------------------
\section{Avaliação da Qualidade}

% ----------------------------------------------------------
% Seção
% ----------------------------------------------------------
%\section{Padrões / Certificações}

% ==========================================================
% CAPITULO
% ==========================================================
\chapter{Qualidade de Software e Modelos de Processo}

% \section{A qualidade nos modelos formais}

% \section{Métodos Ágeis}

% \subsection{TDD - Test Driven Development}

% ==========================================================
% CAPITULO
% ==========================================================
\chapter{Testes}

% ----------------------------------------------------------
% Seção
% ----------------------------------------------------------
\section{Definição}

\begin{citacao}
Por melhores que sejam as técnicas de modelagem e especificação de software, por mais disciplinada e experiente que seja a equipe de desenvolvimento, sempre haverá um fator que faz que o teste de software seja necessário: o erro humano. É um mito pensar que bons desenvolvedores, bem concentrados e com boas ferramentas serão capazes de desenvolver software sem erros \apud{beizer1990}{WAZLAWICK2013}. % origem: página 287 do pdf
\end{citacao}


% % ----------------------------------------------------------
% % Seção
% % ----------------------------------------------------------
% \section{Verificação e Validação}

% Definição lorem ipsum dolor sit amet, consectetur adipisicing elit, sed do eiusmod
% tempor incididunt ut labore et dolore magna aliqua. Ut enim ad minim veniam,
% quis nostrud exercitation ullamco laboris nisi ut aliquip ex ea commodo
% consequat. Duis aute irure dolor in reprehenderit in voluptate velit esse
% cillum dolore eu fugiat nulla pariatur. Excepteur sint occaecat cupidatat non
% proident, sunt in culpa qui officia deserunt mollit anim id est laborum.


% \subsection{Análise estática}

% Definição lorem ipsum dolor sit amet, consectetur adipisicing elit, sed do eiusmod
% tempor incididunt ut labore et dolore magna aliqua. Ut enim ad minim veniam,
% quis nostrud exercitation ullamco laboris nisi ut aliquip ex ea commodo
% consequat. Duis aute irure dolor in reprehenderit in voluptate velit esse
% cillum dolore eu fugiat nulla pariatur. Excepteur sint occaecat cupidatat non
% proident, sunt in culpa qui officia deserunt mollit anim id est laborum.


% \subsection{Inspeções}


% \subsection{Revisões}


% \subsection{Caso de teste}

% ----------------------------------------------------------
% Seção
% ----------------------------------------------------------
\section{Tipos de Problemas}

% FIXME: página 32 de Software Product Quality Control

\begin{itemize}
    \item Falha (\emph{Failure / problem})
    \item Defeito (\emph{Fault / defect / bug})
    \item Erro (\emph{Error / mistake})
\end{itemize}

\begin{figure}[h]
    \centering
    \graphicspath{ {./graphics/} }
    \includegraphics[scale=0.8]{defeito_falha_erro}
    \caption{Visão geral de tipos de problemas}
    \label{fig:tipos_problemas}
    \legend{Fonte: \citeonline{wagner2013}}
\end{figure}

% ----------------------------------------------------------
% Seção
% ----------------------------------------------------------
\section{Categorias}
* em desenvolvimento *

\subsection{Estrutural}

\subsection{Funcional}



% ----------------------------------------------------------
% Seção
% ----------------------------------------------------------
\section{Técnicas}
* em desenvolvimento *

\subsection{White-box}

\subsection{Black-box}

\subsection{Grey-box}

% ----------------------------------------------------------
% Seção
% ----------------------------------------------------------
\section{Níveis}
* em desenvolvimento *

\subsection{Unitário / componente}

\subsection{Integração / operação do sistema / funcional}

\subsection{Validação}

\subsection{Sistema}

% ----------------------------------------------------------
% Seção
% ----------------------------------------------------------
\section{Foco do teste}
* em desenvolvimento *

\subsection{Conformidade}

\subsection{Configuração}

\subsection{Recuperação}

\subsection{Regressão}

\subsection{Estresse}

\subsection{Desempenho}

\subsection{Usabilidade}

\subsection{Acessibilidade}

\subsection{Internacionalização}

% ----------------------------------------------------------
% Seção
% ----------------------------------------------------------
\section{Ferramentas}
* em desenvolvimento *

% ----------------------------------------------------------
% Seção
% ----------------------------------------------------------
\section{Fluxo de Teste em Diferentes Metodologias}
* em desenvolvimento *

% ---
% Finaliza a parte no bookmark do PDF, para que se inicie o bookmark na raiz
% ---
\bookmarksetup{startatroot}% 
% ---

% ==========================================================
% CAPITULO
% ==========================================================
\chapter*[Conclusão]{Conclusão}
\addcontentsline{toc}{chapter}{Conclusão}

\emph{*em desenvolvimento*}

%\lipsum[31-33]

% ----------------------------------------------------------
% ELEMENTOS PÓS-TEXTUAIS
% ----------------------------------------------------------
\postextual


% ----------------------------------------------------------
% Referências bibliográficas
% ----------------------------------------------------------
%\bibliography{abntex2-modelo-references}
\bibliography{andrei-bibtex}

% ----------------------------------------------------------
% Glossário
% ----------------------------------------------------------
%
% Consulte o manual da classe abntex2 para orientações sobre o glossário.
%
%\glossary

%---------------------------------------------------------------------
% INDICE REMISSIVO
%---------------------------------------------------------------------

\printindex

\end{document}
