% ------------------------------------------------------------------------
% ------------------------------------------------------------------------
% abnTeX2: Modelo de Trabalho Academico (tese de doutorado, dissertacao de
% mestrado e trabalhos monograficos em geral) em conformidade com 
% ABNT NBR 14724:2011: Informacao e documentacao - Trabalhos academicos -
% Apresentacao
% ------------------------------------------------------------------------
% ------------------------------------------------------------------------

\documentclass[
	% -- opções da classe memoir --
	12pt,				% tamanho da fonte
	openright,			% capítulos começam em pág ímpar (insere página vazia caso preciso)
	oneside,			% para impressão em verso e anverso. Oposto a oneside
	a4paper,			% tamanho do papel. 
	% -- opções da classe abntex2 --
	%chapter=TITLE,		% títulos de capítulos convertidos em letras maiúsculas
	%section=TITLE,		% títulos de seções convertidos em letras maiúsculas
	%subsection=TITLE,	% títulos de subseções convertidos em letras maiúsculas
	%subsubsection=TITLE,% títulos de subsubseções convertidos em letras maiúsculas
	% -- opções do pacote babel --
	english,			% idioma adicional para hifenização
	brazil,				% o último idioma é o principal do documento
	]{abntex2}


% ---
% PACOTES
% ---

% ---
% Pacotes fundamentais 
% ---
\usepackage{cmap}				% Mapear caracteres especiais no PDF
\usepackage{lmodern}			% Usa a fonte Latin Modern			
\usepackage[T1]{fontenc}		% Selecao de codigos de fonte.
\usepackage[utf8]{inputenc}		% Codificacao do documento (conversão automática dos acentos)
\usepackage{lastpage}			% Usado pela Ficha catalográfica
\usepackage{indentfirst}		% Indenta o primeiro parágrafo de cada seção.
\usepackage{color}				% Controle das cores
\usepackage{graphicx}			% Inclusão de gráficos
\usepackage{abntex2-ufc}
% ---
		
% ---
% Pacotes adicionais, usados apenas no âmbito do Modelo Canônico do abnteX2
% ---
\usepackage{lipsum}				% para geração de dummy text
\usepackage{booktabs}
\usepackage[table,xcdraw]{xcolor}
% ---

% ---
% Pacotes de citações
% ---
\usepackage[brazilian,hyperpageref]{backref}	 % Paginas com as citações na bibl
\usepackage[alf]{abntex2cite}	% Citações padrão ABNT

% --- 
% CONFIGURAÇÕES DE PACOTES
% --- 

% ---
% Configurações do pacote backref
% Usado sem a opção hyperpageref de backref
\renewcommand{\backrefpagesname}{Citado na(s) página(s):~}
% Texto padrão antes do número das páginas
\renewcommand{\backref}{}
% Define os textos da citação
 \renewcommand*{\backrefalt}[4]{
 	\ifcase #1 %
 		Nenhuma citação no texto.%
 	\or
 		Citado na página #2.%
 	\else
 		Citado #1 vezes nas páginas #2.%
 	\fi}%
% ---


% ---
% Informações de dados para CAPA e FOLHA DE ROSTO
% ---
\titulo{Modelos de qualidade de produto de software com perspectivas sobre métodos ágeis, inspeções e testes}
\autor{Andrei Bosco Bezerra Torres}
\local{Fortaleza - CE}
\data{Abril de 2015}
\orientador[Orientadores]{Professor: José Marques Soares}
\instituicao{%
  Universidade Federal do Ceará -- UFC
  \par
  Departamento de Engenharia de Teleinformática
  \par
  Programa de Pós-Graduação em Engenharia de Teleinformática}
\tipotrabalho{Atividade de mestrado}
% O preambulo deve conter o tipo do trabalho, o objetivo, 
% o nome da instituição e a área de concentração 
\preambulo{Trabalho da disciplina de Engenharia de Software apresentado como parte das atividades.}
% ---


% ---
% Configurações de aparência do PDF final

% alterando o aspecto da cor azul
\definecolor{blue}{RGB}{41,5,195}

% informações do PDF
\makeatletter
\hypersetup{
     	%pagebackref=true,
		pdftitle={\@title}, 
		pdfauthor={\@author},
    	pdfsubject={\imprimirpreambulo},
	    pdfcreator={LaTeX with abnTeX2},
		pdfkeywords={abnt}{latex}{abntex}{abntex2}{trabalho acadêmico}, 
		colorlinks=true,       		% false: boxed links; true: colored links
    	linkcolor=blue,          	% color of internal links
    	citecolor=blue,        		% color of links to bibliography
    	filecolor=magenta,      		% color of file links
		urlcolor=blue,
		bookmarksdepth=4
}
\makeatother
% --- 

% --- 
% Espaçamentos entre linhas e parágrafos 
% --- 

% O tamanho do parágrafo é dado por:
\setlength{\parindent}{1.3cm}

% Controle do espaçamento entre um parágrafo e outro:
% \setlength{\parskip}{0.2cm}  % tente também \onelineskip
\setlength{\onelineskip}{0.2cm}

% ---
% compila o indice
% ---
\makeindex
% ---

% ----
% Início do documento
% ----
\begin{document}

% Retira espaço extra obsoleto entre as frases.
\frenchspacing 

% ----------------------------------------------------------
% ELEMENTOS PRÉ-TEXTUAIS
% ----------------------------------------------------------
% \pretextual

% ---
% Capa
% ---
\imprimircapa
% ---

% ---
% Folha de rosto
% (o * indica que haverá a ficha bibliográfica)
% ---
%\imprimirfolhaderosto*
\imprimirfolhaderosto
% ---

% ---
% RESUMOS
% ---

% resumo em português
\begin{resumo}
 O foco deste trabalho é abordar sobre qualidade de software e todo o processo envolvido no desenvolvimento e entrega de um software de alto padrão de qualidade, abordando os modelos de qualidade de produto, gestão, métricas e inspeção. Também será abordada a aplicação de modelos de qualidade em projetos que adotam métodos ágeis e como isso pode impactar o resultado final do produto.

 \vspace{\onelineskip}
    
 \noindent
 \textbf{Palavras-chaves}: modelos de qualidade de produto, qualidade de software, inspeção, testes, validação, verificação, métodos ágeis.
\end{resumo}

% ---
% inserir lista de ilustrações
% ---
\pdfbookmark[0]{\listfigurename}{lof}
\listoffigures*
\cleardoublepage
% ---

% ---
% inserir lista de tabelas
% ---
\pdfbookmark[0]{\listtablename}{lot}
\listoftables*
\cleardoublepage
% ---

% ---
% inserir lista de abreviaturas e siglas
% ---
\begin{siglas}
    \item[ABNT] Associação Brasileira de Normas Técnicas
    \item[COQUAMO] \emph{Constructive Quality Model} - Modelo de Qualidade Construtivo
    \item[IEC] \emph{International Electrotechnical Commission} - Comissão Eletrotécnica Internacional
    \item[IEEE] \emph{Institute of Electrical and Electronics Engineers} - Instituto de Engenheiros Eletricistas e Eletrônicos
    \item[ISO] \emph{International Organization for Standardization} - Organização Internacional para Padronização
    \item[QA] \emph{Quality Assurance} - Garantia de Qualidade
    \item[QMOOD] \emph{Quality Model for Object-Oriented Design} - Modelo de Qualidade
para Projeto Orientados a Objeto
    \item[SQuaRE] \emph{Systems and software Quality Requirements and Evaluation} - Avaliação e Requisitos de Qualidade de Sistemas e Software
\end{siglas}
% ---

% ---
% inserir lista de símbolos
% ---
% \begin{simbolos}
%   \item[$ \Gamma $] Letra grega Gama
%   \item[$ \Lambda $] Lambda
%   \item[$ \zeta $] Letra grega minúscula zeta
%   \item[$ \in $] Pertence
% \end{simbolos}
% ---

% ---
% inserir o sumario
% ---
\pdfbookmark[0]{\contentsname}{toc}
\tableofcontents*
\cleardoublepage
% ---



% ----------------------------------------------------------
% ELEMENTOS TEXTUAIS
% ----------------------------------------------------------
\textual

% ==========================================================
% CAPITULO
% ==========================================================
\chapter[Introdução]{Introdução}

Este trabalho faz uma abordagem sobre qualidade de software, iniciando uma contextualização sobre seus diversos tipos de definições e conceitos básicos e, posteriormente, focando nos modelos de qualidade existentes, desde os mais relevantes historicamente até o padrão atual ISO/IEC. Como fechamento será apresentada uma análise sobre como os modelos de processo de software abordam (ou não) a qualidade, com foco nas metodologias ágeis.

% ----------------------------------------------------------
% Seção
% ----------------------------------------------------------
\section{Definição}

Definir \emph{qualidade de software}, ou mesmo \emph{qualidade} não é tão simples como parece. O conceito de qualidade de um objeto está sob discussão por filósofos há mais de 2 mil anos, desde Platão, com sua teoria do belo \cite{wagner2013}. É possível identificar quando um produto é de qualidade, mas normalmente não é fácil indicar o que lhe dá essa característica. No caso de qualidade para Engenharia de Software, vários autores trazem suas definições e existem vários modelos e organizações que almejam criar uma estrutura para a criação de softwares de qualidade.

\begin{description}
    \item[Definição \citeonline{IEEE1990} para Qualidade:] \hfill
        \begin{itemize}
            \item O grau em que um sistema, componente ou processo alcança os requisitos especificados
            \item O grau em que um sistema, componente ou processo atinge as necessidades ou expectativas de um cliente ou usuário
        \end{itemize}
    \item[Definição \citeonline{IEEEvocabulary2010} para Qualidade:] \hfill
        \begin{itemize}
            \item O grau em que um sistema, componente ou processo alcança os requisitos especificados
            \item A habilidade de um produto, serviço, sistema, componente ou processo em atender as necessidades, expectativas ou requisitos do cliente ou usuário
            \item A totalidade das características de uma entidade que afetam sua capacidade de satisfazer necessidades explícitas e implícitas
            \item A conformidade com as expectativas dos usuários, a conformidade com os requisitos do usuário, a satisfação do cliente, confiabilidade e nível de defeitos presentes
            \item O grau em que um conjunto de características inerentes satisfaz os requisitos
            \item O grau em que um sistema, componente ou processo atinge as necessidades ou expectativas de um cliente ou usuário
        \end{itemize}
    \item[Definição NBR ISO/IEC 9126 (\citeyear{nbr9126}) para Qualidade:]\footnote{Substituída pela nova família de Normas NBR ISO/IEC 25000, disponível para aquisição em: \url{http://www.abntcatalogo.com.br/}}
        \begin{itemize}
            \item  Totalidade de características de uma entidade que lhe confere a capacidade de satisfazer as necessidades explícitas e implícitas.
        \end{itemize}
    \item[Definição de \citeonline{PRESSMAN2010} para Qualidade de Software:] \hfill % página 400, 14.2
        \begin{itemize}
            \item Um processo de software eficaz aplicado de uma maneira que cria um produto útil, que oferece valor mensurável para quem produz e para quem usa.
        \end{itemize}
\end{description}

A evolução das definições da IEEE entre 1990 e 2010 torna claro a necessidade de uma grande abrangência de conceitos. Uma definição objetiva é difícil de ser concebida porque é quase impossível analisar objetivamente se um requisito foi cumprido adequadamente ou não. Pode haver divergências de interpretação de um requisito entre o desenvolvedor e o cliente ou pode haver conflito de expectativas entre diferentes \emph{stakeholders}\footnote{Qualquer pessoa ou organização que tenha interesse, ou seja afetado pelo projeto.}. %FIXME: decidir se esse trecho continua ou não: Então em que se deve pensar para chegar à um produto com qualidade?

\section{Fatores de Qualidade}

Devido aos questionamentos levantados anteriormente, cabe à equipe de gerenciamento de qualidade julgar se um nível de qualidade foi alcançado. \citeonline{SOMMERVILLE2011} exemplifica algumas questões que auxiliam a validar a qualidade de um sistema:
\begin{enumerate}
    \item Os padrões de programação e documentação foram seguidos durante o processo de desenvolvimento?
    \item O software foi testado adequadamente?
    \item O software é suficientemente confiável para ser posto em produção?
    \item A performance do software é aceitável para uso normal?
    \item O software é utilizável?
    \item O software é bem estruturado e compreensível?
\end{enumerate}


% REF: página 673 de Software Engineering (9th Edition)
\citeonline{SOMMERVILLE2011} lembra que a análise da qualidade de um software não está sujeita apenas a analisar se uma funcionalidade foi implementada corretamente, mas também depende das características não-funcionais. Se uma funcionalidade não funcionar como esperado, o usuário normalmente encontra outras maneiras de realizar o que deseja, porém, se o software for lento ou instável, ele será impedido de alcançar seus objetivos.

% REF: ver página 430 de Software Engineering A Practitioner's Approach 7th Edition - Roger Pressman
Na visão de \citeonline{GARVIN1987}, a qualidade deve ser analisada de um ponto de vista multidimensional, desde a conformidade com os requisitos até o caráter estético da aplicação (ver \autoref{tab:qualidade_garvin}).

\begin{table}[h]
    \caption{Dimensões de qualidade de Garvin}
    \label{tab:qualidade_garvin}
    \begin{tabular}{p{3.7cm}|p{11cm}}
        \textbf{Performance} & o software entrega todo o conteúdo/funções/característica especificadas nos requisitos e de maneira que agregue valor ao usuário? \\ \hline
        \textbf{Característica} & o software possui alguma característica que surpreenderá e agradará usuários finais novatos? \\ \hline
        \textbf{Confiabilidade} & o software disponibiliza suas funções sem falhas? \\ \hline
        \textbf{Conformidade} & o software está em conformidade com padrões locais e externos que são relevantes à aplicação? Ele segue convenções de código e de design? \\ \hline
        \textbf{Durabilidade} & O software pode ser mantido (alterado) e corrigido (debugado) sem efeitos colaterais indesejáveis? \\ \hline
        \textbf{Manutenibilidade} & o software pode ser mantido (alterado) e corrigido (debugado) em um espaço de tempo aceitável? \\ \hline
        \textbf{Estética} & o valor estético, elegância, a \emph{presença} do software.\\ \hline
        \textbf{Percepção} & Preconceitos (positivos ou negativos) que influenciam a percepção de qualidade. Um software anteriormente produzido por um desenvolvedor que produziu algo de baixa qualidade causará uma influência negativa na nova versão.\\
  \end{tabular}
  \legend{Fonte: \citeonline{PRESSMAN2010}}
\end{table}

\citeonline{SOMMERVILLE2011} reforça que é impossível um sistema ser totalmente otimizado em todos os aspectos, portanto é necessário ser analisado e definido quais atributos são vitais para a aplicação. Caso a performance seja crítica, então as outras características podem ser sacrificadas, por exemplo.

Conforme dito anteriormente, a qualidade é um conceito extremamente amplo e abrangente, e nem sempre simples de ser avaliado objetivamente. Com isso, é necessário metas concretas, um modelo com o objetivo de descrever, avaliar e/ou prever qualidade \cite{wagner2013}. Diversos modelos serão descritos e analisados no capítulo \ref{cap:modelos_qualidade}.

% ----------------------------------------------------------
% Seção
% ----------------------------------------------------------
\section{Tipos de Qualidade}
% REF: página 21 do livro Software Product Quality Control

A qualidade de software é dividida em dois aspectos: a \emph{qualidade do produto} em si e a \emph{qualidade do processo}. A qualidade do produto foca no controle de qualidade para gerar um produto final que atenda aos requisitos e traga satisfação do usuário, enquanto a qualidade do processo foca no controle da produção para otimizar o desenvolvimento e o uso otimizado dos recursos disponíveis.

Os modelos de maturidade, que tem como foco a \emph{qualidade do processo}, não serão abordados, tendo este trabalho a \emph{qualidade do produto} e os modelos de qualidade criados para ele como foco principal. Para uma abordagem sobre modelos de qualidade de processo recomendo a leitura do capítulo 12 de \citeonline{WAZLAWICK2013}.

\citeonline{WAZLAWICK2013} indica que apesar de não existir garantia que um bom processo vá resultar em um bom produto, em geral admite-se que a mesma equipe com um bom processo vá produzir produtos melhores do que se não tivesse processo algum.

% fonte: http://www.bayt.com/en/specialties/q/52980/can-you-differentiate-between-product-quality-and-process-quality/
% Product quality is focusing on meeting tolerances in the end result of the manufacturing activities. The end result is measured on a standard of "good enough".
% Process quality focuses on each activity and forces the activities to achieve maximum tolerances irrespective of the end result. 
% For product quality we use quality control to assure product confirm to production standards and tolerance where for process quality we use quality assurance to assure process effectiveness and efficiency 

% REF: página 228 de Engenharia de Software - conceitos e práticas
% REF: berander2005, página 3
% REF: pressman2010, página 428, final da página

% FIXME: vale manter essas definições, já que não são exploradas posteriormente?
E dentro do contexto de qualidade do produto, \citeonline{PRESSMAN2010} define dois tipos de qualidade:

\label{tipos_qualidades}

\begin{description}
    \item [Qualidade de projeto (atender as necessidades do cliente):] qualidade identificável através de características não mesuráveis objetivamente, a qualidade do produto em atender as expectativas do cliente, explícitas ou não.
    \item [Qualidade de conformação (conformidade com as especificações):] avalia quão \\ bem o produto atende aos requisitos, em conformidade com as especificações definidas, através de características mesuráveis.
\end{description}

% ----------------------------------------------------------
% Seção
% ----------------------------------------------------------
\section{Custo da qualidade}
\begin{citacao} % pagina 435 - pressman
Se você produzir um sistema de software de péssima qualidade, então você perderá pois ninguém irá comprá-lo. Por outro lado, se você gastar tempo infinito, extremo esforço e grandes quantidades de dinheiro para produzir um software absolutamente perfeito então você demorará tanto para concluí-lo e será tão dispendioso que você acabará fora do mercado do mesmo jeito. Ou você terá perdido a janela de oportunidade ou terá exaurido sua verba. Então as pessoas da indústria tentam encontrar o posicionamento mágico onde o produto é bom o suficiente para não ser rejeitado de cara, como durante a avaliação, e sem ser preso a um perfeccionismo e excesso de trabalho que faça com ele acabe demorando ou saindo caro demais. (MEYER, Bertrand apud \citeauthoronline{PRESSMAN2010}, 2010)
\end{citacao}

% REF: pressman, 436

A citação acima apresenta o \emph{dilema da qualidade}, explorado por \citeonline{PRESSMAN2010}, que discute que apesar de ser válido que os engenheiros de software busquem a produção de um sistema de alta qualidade, a vida real apresenta limitações na execução de um projeto.

Tomando a citação de Meyer como base, um software ``bom o suficiente'' seria um software que executa bem suas funções mas que possui limitações/bugs conhecidos, e espera-se que o usuário ignore tais falhas. \citeonline{PRESSMAN2010} alerta que este é um caminho perigoso, pois um produto com bugs pode manchar a reputação de uma empresa (ou até mesmo quebrá-la), e que em alguns mercados tal pensamento pode até mesmo ser ilegal, como na área da medicina ou aviação, quando vidas humanas dependem do funcionamento correto.

Um fator que normalmente pesa contra a implementação de um processo de qualidade é o investimento, de tempo e de dinheiro, existindo uma dificuldade de se ver um retorno justificável. \citeonline{PRESSMAN2010} concorda que gerar um produto de qualidade gera um custo, mas lembra que gerar um produto sem qualidade também. É necessário compreender o custo envolvido e tomar decisões de maneira consciente. O custo da qualidade pode ser dividido em 3 tipos:

\begin{itemize}
    \item \textbf{Custo de prevenção}: o custo das atividades gerenciais para planejar e coordenar todas as atividades de controle de qualidade, o custo adicional das atividades técnicas, e custo de treinamento associado a essas atividades.
    \item \textbf{Custo de avaliação}: o custo de conduzir revisões técnicas, coleta de dados, custo de testes e debugs.
    \item \textbf{Custo de falha}: os custos que sumiriam caso nenhuma falha surgisse antes e após a entrega do produto. Os custos relacionados à falhas podem ser categorizados em:
    \begin{itemize}
        \item \textbf{Custo interno de falhas}: custo para reparar um erro, custo de efeitos colaterais causados pelo erro.
        \item \textbf{Custo externo de falhas}: resolução do problema, devolução do produto, contato p/ suporte, e reputação ruim
    \end{itemize}
\end{itemize}

\citeonline{PRESSMAN2010} apresenta um gráfico (ver \autoref{fig:pressman-grafico_custos_qualidade}) baseado em dados de uma pesquisa realizada em 2001, que ilustra que quanto mais tardiamente uma falha é encontrada maior o custo para sua resolução.

\begin{figure}[h]
    \centering
    \caption{Custo de um erro em diferentes fases de um projeto}
    \graphicspath{ {./graphics/} }
    \includegraphics[scale=0.8]{pressman-grafico_custos_qualidade}
    \label{fig:pressman-grafico_custos_qualidade}
    \legend{Fonte: \citeonline{PRESSMAN2010}}
\end{figure}

% ==========================================================
% CAPITULO
% ==========================================================
\chapter{Modelos de Qualidade de Produto}
\label{cap:modelos_qualidade}

%FIXME: fazer introdução ao capítulo?

% ----------------------------------------------------------
% Seção
% ----------------------------------------------------------
\section{Definição}

Um \emph{Modelo de Qualidade} é um modelo cujo objetivo é descrever, avaliar e/ou prever qualidade. Muitos modelos de qualidade fazem uma decomposição da qualidade geral de um produto em sub-qualidades para que se torne mais fácil de compreender e gerenciar \cite{wagner2013}.

Este tópico vem sendo pesquisado à décadas e existe um grande número de modelos \cite{Klas2009} e seria inviável analisar todos, então este trabalho focará nos modelos mais relevantes, seja por sua importância histórica ou por ser um padrão internacional. \citeonline{wagner2013} propõe uma divisão entre os modelos de qualidade, que será utilizada como referência:
\begin{itemize}
    \item \emph{Hierárquicos}: Um modelo de decomposição hierárquica dos fatores de qualidade. Ex.: McCall, Boehm, ISO 9126;
    \item \emph{Meta-modelos}: Um modelo que define como modelos de qualidade válidos devem ser estruturados, cada um com propriedades e métricas específicas para o projeto. Ex.: Dromey, COQUAMO (\emph{Constructive Quality Model} - Modelo de Qualidade Construtivo), QMOOD (\emph{Quality Model for Object-Oriented Design} - Modelo de Qualidade para Projeto Orientados a Objeto);
    \item \emph{Estatístico ou implícito}: Um modelo estatístico que busca estimar ou prever os fatores de qualidade, através de ferramentas de análise ou de testes. Ex.: Modelo de confiabilidade de crescimento (\emph{reliability growth model}).
\end{itemize}

% ----------------------------------------------------------
% Seção
% ----------------------------------------------------------
\section{Modelos Hierárquicos}
\subsection{Modelo McCall (1977)}
% foco na qualidade do produto, dividido em visão externa (fatores de qualidade a especificar) e visão interna (critérios de qualidade para desenvolver)

Um dos predecessores mais renomado dos modelos de qualidade atuais é o modelo de \citeonline{general1977factors}, criado para o exército dos EUA e focado nos desenvolvedores de sistema e no processo de desenvolvimento. Este modelo define fatores de qualidade que tentam aproximar o ponto de vista do usuário e as prioridades dos desenvolvedores. Estes fatores são divididos em três perspectivas (ver \autoref{fig:mcCall-fatores_qualidade}):

\begin{itemize}
    \item \emph{Revisão do produto} (capacidade em sofrer alterações): inclui manutenibilidade (o esforço necessário para localizar e corrigir um erro), flexibilidade (facilidade de fazer alterações necessárias no produto) e testabilidade (a facilidade de testar o produto, garantindo que não possui erros e está de acordo com as especificações);
    \item \emph{Transição do produto} (adaptabilidade a novos ambientes): é focado na portabilidade (o esforço necessário para transferir o produto de um ambiente para outro), reusabilidade (a facilidade de reusar o produto em outro contexto) e interoperabilidade (o esforço de conectar o produto com outro);
    \item \emph{Operação do produto} (características operacionais): a qualidade operacional do produto depende da exatidão (o grau em que o produto segue as especificações), confiabilidade (a habilidade do produto em não falhar), eficiência (eficiência de execução, de armazenamento, do uso de recursos), integridade (proteção contra acessos não autorizados) e usabilidade (a facilidade de uso).
\end{itemize}

\begin{figure}[h]
    \centering
    \caption{Perspectivas e fatores de qualidade de McCall}
    \graphicspath{ {./graphics/} }
    \includegraphics[scale=0.8]{mcCall-fatores_qualidade}
    \label{fig:mcCall-fatores_qualidade}
    \legend{Fonte: \citeonline{PRESSMAN2010}}
\end{figure}

% REF: página 7 de software_quality_atributes
McCall define uma hierarquia (\autoref{fig:mcCall-quality_model}) relacionando os fatores citados acima com critérios:
\begin{itemize}
    \item 11 fatores (especificação): descrever a visão externa do software, do ponto de vista do usuário;
    \item 23 critérios de qualidade (construção): descrever a visão interna do software, do ponto de vista dos desenvolvedores;
    \item Métricas (controle): definidas e utilizados para prover método de medida.
\end{itemize}

\begin{figure}[H]
    \centering
    \caption{Hierarquia de fatores e critérios de McCall}
    \graphicspath{ {./graphics/} }
    \includegraphics[scale=0.85]{mcCall-graph-andrei}
    \label{fig:mcCall-quality_model}
    \legend{Fonte: \citeonline{Naik2011} [Adaptação do autor]}
\end{figure}

Os fatores de qualidade retratam as características comportamentais do sistema, os critérios são atributos para um ou mais fatores, enquanto as métricas buscam capturar os aspectos do critério de qualidade \cite{berander2005}.

\citeonline{Naik2011} demonstram como definir as métricas:
\begin{itemize}
    \item Formular um conjunto de questões sobre o critério de qualidade que possam ser respondidas com \emph{sim} ou \emph{não};
    \item Dividir as respostas \emph{sim} pelo número total de questões para obter um valor entre 0 e 1. Esse valor será a métrica do atributo.
\end{itemize}

O ponto negativo desta proposta é a subjetividade de avaliação da métrica, que cabe ao julgamento da pessoa que respondeu as perguntas \cite{berander2005}. \citeonline{Naik2011} julgam que alguns critérios podem levar a perguntas que não sejam facilmente respondidas apenas com 'sim' ou 'não. Eles citam os seguintes exemplos:
\begin{itemize}
    \item \emph{Toda a documentação está escrita de maneira clara e concisa, de maneira que todos os procedimentos, funções e algoritmos sejam facilmente compreendidos?} - métrica do critério \textbf{auto-descritivo}, possível de se responder com 'sim' ou 'não'.
    \item \emph{O design do software é simples?} - métrica do critério \textbf{operabilidade}, o conceito de ``simplicidade'' é subjetivo, e uma escala de valores possíveis funcionaria melhor do que resposta `sim' ou `não'.
\end{itemize}

% ----------------------------------------------------------
% Seção
% ----------------------------------------------------------
\subsection{Modelo Boehm (1978)}
% três níveis de características: alto nível (usuário), intermediária (software) e primitiva (métricas e avaliação)
% FIXME: indicar que ele criou o modelo em espiral?

Outro modelo histórico foi criado por \citeonline{BOEHM1978}, que, de maneira similar ao modelo de McCall, apresenta um modelo hierárquico de características. Para Boehm, o primeiro fator a ser analisado é a utilidade geral (\emph{General utility}) do software, pois se ele não for realmente útil então todo o projeto é uma perda de tempo e dinheiro \cite{PFLEEGER2010}.

O modelo de Boehm é dividido em três níveis, onde cada um contribui para a qualidade geral: \emph{alto nível} (usuário), \emph{intermediário} (software) e \emph{primitivas} (métricas). As características de alto nível tem como objetivo responder a três questões quanto ao uso do software pelo usuário, que não é necessariamente o cliente \cite{berander2005}:

% REF: página 614, pfleeger - 'Software Engineering Theory and Practice 4e Shari Pfleeger Joanne Atlee 9780136061694.pdf'
\begin{itemize}
    \item Utilidade `como está' (\emph{As is utility}) - usuário final: quão fácil (eficiência, confiança, usabilidade) é de utilizá-lo tal 'como está' no momento?
    \item Portabilidade - desenvolvedor: posso continuar a utilizá-lo caso altere o meu ambiente? Caso o compilador seja substituído, o programa não deve sofrer degradação de funcionalidades, por exemplo.
    \item Manutenibilidade - desenvolvedor mantenedor: quão fácil é entender, modificar e testar?
\end{itemize}

Já as características de nível médio representam os 7 fatores de qualidade esperados pelos desenvolvedores (ver \autoref{fig:boehm-hierarquia_modelo}), enquanto o nível de características primitivas provê as métricas de avaliação do sistema. Uma das principais diferenças deste modelo para o de McCall é que este abrange a performance do hardware como característica relevante à qualidade. Ou seja, para Boehm a qualidade do software deve satisfazer tanto as necessidades do usuário final quanto os desenvolvedores envolvidos, e como reflexo dessa qualidade o software deve: fazer o que o usuário requer; utilizar os recursos computacionais de maneira correta e eficaz; ser fácil de aprender e utilizar; e ser bem projetado, bem programado, facilmente testável e de fácil manutenção \cite{PFLEEGER2010}.

Porém, segundo \citeonline{SURYN2014}, apesar deste modelo ser um avanço comparado ao modelo de McCall, as características de utilidade geral (\emph{general utility}) e 'utilitário como está' (\emph{as-is utility}) ainda são conceitos genéricos e imprecisos, resultando em uma aplicabilidade limitada.

\begin{figure}[H]
    \centering
    \caption{Hierarquia do Modelo de Qualidade de Boehm}
    \graphicspath{ {./graphics/} }
    \includegraphics[scale=0.95]{boehm-hierarquia_modelo-andrei}
    \label{fig:boehm-hierarquia_modelo}
    \legend{Fonte: \citeonline{PFLEEGER2010} [Adaptação do autor]}
\end{figure}

% ----------------------------------------------------------
% Seção
% ----------------------------------------------------------
\subsection{Padrão ISO 9126 (1991)}
O padrão ISO 9126 (\emph{Software Product Evaluation: Quality Characteristics and Guidelines for their Use} - Avaliação do Produto de Software: Características de Qualidade e Diretrizes para sua Utilização) foi criado com base nos modelos de McCall e Boehm, focando em três aspectos da qualidade de software \cite{SURYN2014}:
\begin{description}
    \item [Qualidade no uso:] qualidade de uso é o ponto de vista do usuário quanto a qualidade do software ao ser utilizado em um ambiente e contexto específico. Ele avalia se o usuário consegue alcançar seus objetivos, e não as propriedades do software.
    \item [Qualidade externa:] características do software de um ponto de vista externo. A qualidade do software ao ser executado, medido e avaliado em um ambiente simulado ou com dados simulados, utilizando métricas externas.
    \item [Qualidade interna:] a totalidade dos atributos do software que determinam sua habilidade em satisfazer as necessidades implícitas e explícitas do usuário, em condições específicas.
\end{description}

Os fatores de qualidade interna e externas são divididos em uma hierarquia de 3 níveis, composto de características de qualidade, sub-características e métricas (ver \autoref{fig:iso9126-hierarquia_interno_externo} e \autoref{tab:iso9126-caracteristicas}). Mais de 200 métricas\footnote{Listagem de métricas do padrão estão elencados na ISO/IEC 9126-4, disponível para aquisição em: \url{http://www.abntcatalogo.com.br/}} foram propostas como parte do padrão e outras métricas podem ser criadas e utilizadas de acordo com a necessidade \cite{SURYN2014}.

Já os fatores de qualidade de uso são organizados em uma hierarquia de 2 níveis, composto das características e das métricas de qualidade  (ver \autoref{fig:iso9126-hierarquia_qualidade_uso}). A junção e conexão dos três aspectos da qualidade de software do padrão ISO 9126 formam um modelo preditivo (ver \autoref{fig:iso9126-hierarquia_relacionamento}).

Em comparação com os modelo de McCall e de Boehm, o padrão ISO 9126 apresenta as seguintes diferenças:
\begin{itemize}
    \item O padrão ISO 9126 e o modelo de Boehm abordam as características visíveis ao usuário, enquanto o modelo de McCall foca nas características internas;
    \item Um critério de qualidade no padrão ISO 9126 impacta apenas um fator, enquanto nos modelos de McCall e Boehm um critério pode impactar múltiplos fatores;
    \item Um fator de alto nível nos Modelos de McCall e de Boehm pode ser um fator de baixo nível no padrão ISO 9126 (ex.: \emph{testabilidade}, que é um fator no modelo de McCall e um critério dentro de \emph{manutibilidade} no padrão ISO 9126).
\end{itemize}

\begin{figure}[H]
    \centering
    \caption{Hierarquia de qualidade interna e externa do padrão ISO 9126}
    \graphicspath{ {./graphics/} }
    \includegraphics[scale=1.0]{iso9126-hierarquia_interno_externo-andrei}
    \label{fig:iso9126-hierarquia_interno_externo}
    \legend{Fonte: \citeonline{SURYN2014} [Adaptação do autor]}
\end{figure}

\begin{figure}[h]
    \centering
    \caption{Hierarquia de qualidade de uso do padrão ISO 9126}
    \graphicspath{ {./graphics/} }
    \includegraphics[scale=0.95]{iso9126-hierarquia_qualidade_uso-andrei}
    \label{fig:iso9126-hierarquia_qualidade_uso}
    \legend{Fonte: \citeonline{nbr9126} [Adaptação do autor]}
\end{figure}

\begin{figure}[h]
    \centering
    \caption{Relacionamento entre aspectos de qualidade do padrão ISO 9126}
    \graphicspath{ {./graphics/} }
    \includegraphics[scale=0.95]{iso9126-hierarquia_relacionamento-andrei}
    \label{fig:iso9126-hierarquia_relacionamento}
    \legend{Fonte: \citeonline{nbr9126} [Adaptação do autor]}
\end{figure}

\begin{table}[H]
    \centering
    \caption{Características de qualidade ISO 9126}
    \label{tab:iso9126-caracteristicas}
    \small % tamanho do texto - ref: http://www.sascha-frank.com/latex-font-size.html
    \begin{tabular}{p{3cm} p{12.0cm}}
        \toprule
        \textbf{Característica} & \textbf{Definição} \\ \midrule
        Funcionalidade & Um conjunto de atributos que evidenciam a existência de um conjunto de funções e suas propriedades especificadas. As funções são aqueles que satisfazem as necessidades declaradas ou implícitas \\
        \rowcolor[HTML]{EFEFEF}
        Confiabilidade & Um conjunto de atributos que afetam a capacidade de software de manter seu nível de desempenho sob condições estabelecidas por um período de tempo determinado. \\
        Usabilidade & Um conjunto de atributos que evidenciam o esforço necessário para o uso e à apreciação individual de tal uso por um conjunto de usuários explícito ou implícito. \\
        \rowcolor[HTML]{EFEFEF}
        Eficiência & Um conjunto de atributos que afetam a relação entre o desempenho do software e da quantidade de recursos utilizados sob condições estabelecidas. \\
        Manutibilidade & Um conjunto de atributos que evidenciam o esforço necessário para fazer modificações especificadas (que podem incluir correções, melhorias ou adaptações do software às mudanças no ambiente, mudanças nos requisitos e especificações funcionais). \\
        \rowcolor[HTML]{EFEFEF}
        Portabilidade & Um conjunto de atributos que afetam a capacidade do software de ser transferido de um ambiente para outro (incluindo ambiente organizacional, hardware ou software). \\ \bottomrule
    \end{tabular}
    \legend{Fonte: \citeonline{PFLEEGER2010} [Tradução do autor]}
\end{table}

% ----------------------------------------------------------
% Seção
% ----------------------------------------------------------
\subsection{SQuaRE - ISO/IEC 25010 (\emph{Systems and software Quality Requirements
and Evaluation} - 2011)}
% FONTE: página 230 de Engenharia de Software - Conceitos e Práticas

Em 2011, a família de normas ISO/IEC 25000 substituíram a norma ISO 9126, sendo um dos fatores que levou a substituição o fato de que a antiga norma aplicava-se apenas no processo de desenvolvimento e uso do produto de software, e não se preocupava quanto à definição do produto em si.

O novo modelo de qualidade de produto propõe duas perspectivas, cada uma com um modelo específico (ver \autoref{fig:iso25010-hierarquia_qualidade_completa}): a primeira perspectiva é relacionada ao uso do software, com um modelo chamado \emph{modelo de qualidade no uso}, e a segunda perspectiva é relacionada ao software em si, com um modelo chamado \emph{modelo de qualidade do produto de sistema/software}. O modelo de qualidade no uso é composto por 5 características e 11 métricas. O modelo de qualidade do produto é composto de 8 características, que são divididas em sub-características que se relacionam com as propriedades estáticas e dinâmicas do software.

Em comparação ao modelo de qualidades internas e externas do padrão ISO 9126, o modelo de qualidade do produto do padrão ISO 25010 utiliza terminologias diferentes mas que possuem significados similares \cite{SURYN2014}:

\begin{description}
    \item [Estática:] corresponde à qualidade interna da ISO 9126, que se aplica ao software em desenvolvimento, que não está sendo executado;
    \item [Dinâmico:] corresponde à qualidade externa da ISO 9126, que se aplica ao sistema e software em execução, mas não dentro do contexto operacional (para isto existe o modelo de qualidade no uso).
\end{description}

\citeonline{wagner2013} critica que o padrão não indica quando utilizar cada modelo, e que aparentemente as empresas adotam o modelo de qualidade em uso apenas quando analisam a usabilidade do software. Ele lembra também que esse modelo é uma taxonomia, e não é a única, e talvez nem a melhor, maneira de estruturar as características de qualidade do software, e que se o modelo for utilizado como algo além de um \emph{checklist} então não será o mais adequado.

\begin{figure}[H]
    \centering
    \caption{Hierarquias de qualidade do padrão ISO 25010}
    \graphicspath{ {./graphics/} }
    \includegraphics[scale=0.9]{iso25010-hierarquia_qualidade_completa-andrei-rev2}
    \label{fig:iso25010-hierarquia_qualidade_completa}
    \legend{Fonte: \citeonline{SURYN2014} [Adaptação do autor]}
\end{figure}

% ----------------------------------------------------------
% Seção
% ----------------------------------------------------------
\subsection{Análise dos modelos hierárquicos}

Os modelos apresentados são úteis para auxiliar na análise da qualidade do software desenvolvido, porém nenhum dos modelos explica a lógica de porque algumas características foram consideradas e outras não, porque um atributo está em uma posição da hierarquia, porque nenhum dos modelos aborda segurança, dentre outras questões \cite{PFLEEGER2010}.

\citeonline{wagner2013} também critica a ambiguidade das características de qualidade e a dificuldade da realização de uma medição adequada. Uma pesquisa realizada por \citeonline{wagner2012} apontou que menos de 28\% das empresas pesquisadas implementam tais modelos hierárquicos, e que 71\% delas desenvolveram uma variante própria, o que demonstra uma necessidade de customização.

\citeonline{PRESSMAN2010} concorda com estas críticas, mas lembra que esses modelos proveem uma boa base para medições indiretas e como um checklist para avaliar a qualidade de um sistema.

A \autoref{tab:modelos_comparacao} apresenta uma comparação entre os modelos hierárquicos apresentados, onde é possível notar o crescimento da abrangência de fatores. Alguns fatores deixaram ser ser primários e passaram a ser secundários, portanto foi indicado na tabela por qual fator primário ele foi englobado.

\begin{table}[H]
\centering
    \caption{Comparação entre critérios/objetivos dos modelos de qualidade de McCall, Boehm e ISO 9126}
    \label{tab:modelos_comparacao}
    \begin{tabular}{@{}lllll@{}}
        \toprule
        \textbf{Critérios/Objetivos} & \multicolumn{1}{c}{McCall} & Bohem & ISO 9126 & ISO 25010\\ \midrule
        Exatidão & * & confiabilidade & funcionalidade & funcionalidade \\
        \rowcolor[HTML]{EFEFEF} 
        Confiabilidade & * & * & * & * \\
        Integridade & * &  &  & segurança \\
        \rowcolor[HTML]{EFEFEF} 
        Usabilidade & * & compreensibilidade & * & * \\
        Eficiência & * & * & * & * \\
        \rowcolor[HTML]{EFEFEF} 
        Manutibilidade & * & * & * & * \\
        Testabilidade & * &  & manutibilidade & manutibilidade \\
        \rowcolor[HTML]{EFEFEF} 
        Interoperabilidade & * &  &  & compatibilidade \\
        Flexibilidade & * &  &  & manutibilidade \\
        \rowcolor[HTML]{EFEFEF} 
        Reusabilidade & * &  &  & manutibilidade\\
        Portabilidade & * & * & * & * \\
        \rowcolor[HTML]{EFEFEF} 
        Clareza &  & * &  & \\
        Modificabilidade &  & * & manutibilidade & manutibilidade \\
        \rowcolor[HTML]{EFEFEF} 
        Documentação &  & * &  & \\
        Resiliência &  & * &  & confiabilidade \\
        \rowcolor[HTML]{EFEFEF} 
        %Compreensibilidade &  & * &  & usabilidade \\ % remover?
        Validez &  & * & funcionalidade & funcionalidade \\
        Funcionalidade &  &  & * & * \\
        \rowcolor[HTML]{EFEFEF} 
        Generalidade &  & * &  & \\
        Segurança & & & funcionalidade & * \\ \bottomrule
    \end{tabular}
    \legend{Fonte: \citeonline{berander2005} [Adaptação do autor]}
\end{table}

% ----------------------------------------------------------
% Seção
% ----------------------------------------------------------
\section{Meta-modelos}
\subsection{Modelo Dromey (1995)}
% três princípios fundamentais: atributos de qualidade, propriedades do produto, e as ligações entre eles

O modelo de \citeonline{Dromey1995} traz uma diferente abordagem de McCall e Boehm, pois propõe que o modelo de qualidade seja baseado na perspectiva do produto, na qual a avaliação e características de qualidade irão variar para cada produto \cite{SURYN2014}:

\begin{citacao}
O que deve ser compreendido em qualquer tentativa de criar um modelo de qualidade é que um software não manifesta diretamente os atributos de qualidade. No entanto, ele exibe características de produto que implicam ou contribuem para atributos de qualidade ou outras características (defeitos do produto) que detraem os atributos de qualidade do produto. A maioria dos modelos de qualidade de software falham em lidar adequadamente com a faceta de problema nas características do produto, e eles também falham em relacionar diretamente os atributos de qualidade e as características correspondentes no produto. [Tradução do autor] \cite{Dromey1995}
\end{citacao}

Seguindo esta lógica, Dromey criou um \emph{framework} de avaliação de qualidade que analisa a qualidade de componentes através de propriedade tangíveis. Cada artefato produzido durante a criação do software (ex.: código, documentação, guia de uso, etc) pode ser associado a um modelo de qualidade diferente (uma variável pode ser um componente do modelo de implementação, e um módulo pode ser um componente do modelo de projeto, por exemplo) \cite{SURYN2014}. E as propriedades tangíveis são divididas em quatro tipo de propriedades (ver \autoref{fig:dromey-estrutura_modelo}):

\begin{itemize}
    \item exatidão: verifica se algum princípio básico foi violado
    \item interno: quão bem um componente foi implementado de acordo com o seu uso pretendido
    \item contextual: lida com influências externas e o uso do componente
    \item descritiva: quão bem descrito é o componente
\end{itemize}

Essas propriedades são utilizadas para a avaliar qualidade dos componentes, e para Dromey, um software de alta qualidade é o resultado do somatório de componentes de alta qualidade, desde os requisitos individuais até a nomenclatura das variáveis do código. O modelo de exemplo ilustrado na \autoref{fig:dromey-estrutura_modelo_exemplo} foi criado seguindo os seguintes passos \cite{PFLEEGER2010}:

\begin{enumerate}
    \item identificar os atributos de qualidade de alto nível
    \item identificar os componentes do produto
    \item identificar e classificar as propriedades mais significantes e tangíveis para cada componente
    \item propor um conjunto de axiomas conectando as propriedades do produto com os atributos de qualidade
    \item avaliar o modelo, identificando suas fraquezas, refinando-o e recriando-o
\end{enumerate}

Contudo, \citeonline{SURYN2014} faz uma analogia interessante: utilizar farinha, maçã e canela de alta qualidade resultará uma torta de maçã de alta qualidade? Óbvio que não, outros elementos são necessários: uma receita (arquitetura e processo de execução), as preferências do usuário (fator totalmente ignorado por Dromey) e pessoas com as qualificações e ferramentas para executar a receita apropriadamente. Certamente este modelo pode ser classificado como uma abordagem \emph{bottom-to-top} de qualidade de software.

\begin{figure}[H]
    \centering
    \caption{Estrutura do Modelo de Qualidade de Dromey}
    \graphicspath{ {./graphics/} }
    \includegraphics[scale=0.85]{dromey-estrutura_modelo-andrei}
    \label{fig:dromey-estrutura_modelo}
    \legend{Fonte: \citeonline{Deissenboeck2009} [Adaptação do autor]}
\end{figure}

\begin{figure}[H]
    \centering
    \caption{Avaliação de qualidade de um componente de variável}
    \graphicspath{ {./graphics/} }
    \includegraphics[scale=0.85]{dromey-estrutura_modelo_exemplo-andrei}
    \label{fig:dromey-estrutura_modelo_exemplo}
    \legend{Fonte: \citeonline{SURYN2014} [Adaptação do autor]}
\end{figure}

% ----------------------------------------------------------
% Seção
% ----------------------------------------------------------
% \subsection{COQUAMO}
% FIXME: página 31, suryn (software product quality model - marcado)
%  They see the quality factor as central in a quality model and argue that each factor should be evaluated differently throughout different development phases and, therefore, have different metrics. They also differentiate between application-specific and general qualities. For example, they see reliability as important for any software system, but security is application specific. COQUAMO aimed strongly at establishing quantitative relationships between quality drivers measured by metrics and factors

% ----------------------------------------------------------
% Seção
% ----------------------------------------------------------
\section{Modelos estatísticos}
\subsection{Modelo de confiabilidade de crescimento}

% ----------------------------------------------------------
% Seção
% ----------------------------------------------------------
\section{Aplicabilidade}
Há uma dificuldade de encontrar literatura, seja ela formal (como artigos científicos e livros), ou informal (como artigos online ou blogs), sobre a aplicação dos modelos citados anteriormente em projetos reais. Pode-se inferir que sua \emph{utilidade real} se encontra em projetos de grande porte de grandes corporações e/ou governamentais, onde é necessária uma prestação de contas formal e detalhada sobre o produto de software entregue, e uma certificação de qualidade tem grande peso.

Porém nada impede a utilização dos modelos de qualidade de produto citados em um processo ágil para validar seus atributos de qualidade, porém o modelo escolhido e os critérios teriam que ser adaptados de acordo com a necessidade e iteração do projeto. No fim das contas o modelo funcionaria como um \emph{checklist} para auxiliar a avaliação do testador.

% ==========================================================
% CAPITULO
% ==========================================================
\chapter{Garantia de Qualidade e Testes}

% ----------------------------------------------------------
% Seção
% ----------------------------------------------------------
\section{Definição}
A gestão da qualidade e suas técnicas de garantia de qualidade (\emph{Quality Assurance} - QA) foram desenvolvidas para, em conjunto com novas tecnologias de desenvolvimento e testes, resolver o problema da baixa qualidade dos software que permeou o séc. XX \cite{SOMMERVILLE2011}. QA se refere à todas as atividades envolvidas na melhoria de qualidade do produto, incluindo o treinamento e preparação de toda a equipe \cite{tsui2013}.

\begin{description}
    \item[Definição \citeonline{IEEE1990} para Garantia de Qualidade (QA):] \hfill
        \begin{itemize}
            \item Um padrão planejado e sistemático de todas as ações necessárias para garantir uma confiança adequada de que um item ou produto está em conformidade com os requisitos técnicos estabelecidos.
            \item Um conjunto de atividades destinadas a avaliar o processo pelo qual os produtos são desenvolvidos ou produzidos.
        \end{itemize}
\end{description}

A visão formal sobre o processo de QA é que ela pode ser efetuada pelo gerente do projeto, mas preferencialmente deve ser realizada por um gerente e/ou equipe especializada para que possa ter uma visão objetiva e independente da equipe de desenvolvimento \cite{SOMMERVILLE2011,WAZLAWICK2013}, porém devemos lembrar que essa lógica não se aplica nas metodologias ágeis, onde a responsabilidade de QA recai sobre toda a equipe (\emph{whole-team approach} - abordagem de time coeso).

A função de QA é prover a gerência e equipe técnica com dados sobre a qualidade do produto que está sendo gerado \cite{SOMMERVILLE2011}, e no caso de uma equipe ágil o testador (\emph{tester}) deve garantir que as técnicas de teste (criação dos testes, automatização, histórias do usuário, etc) estejam sendo seguidas corretamente.

\section{Verificação e Validação (V\&V)}
Conforme vimos na \autoref{tipos_qualidades}, podemos dividir a qualidade em dois tipos: \textbf{qualidade de conformação} (conformidade com as especificações), e \textbf{qualidade de projeto} (atender as necessidades do cliente), e considerando estas duas noções existem duas atividades de QA, também conhecidas como V\&V \cite{tsui2013, PRESSMAN2010}:
\begin{itemize}
    \item \textbf{Verificação}: o ato de verificar a conformidade com os requisitos e especificações (você está construindo corretamente o produto?);
    \item \textbf{Validação}: o ato de conferir se o produto concluído atinge os requisitos, especificações e necessidades do usuário (você está construindo o produto correto?).
\end{itemize}

Já as técnicas utilizadas no processo de QA podem ser categorizadas em dois grupos \cite{Huo:2004:SQA:1025117.1025549,Naik2011}\footnote{Para maiores detalhes sobre cada uma das técnicas de QA confira o anexo bla}: % FIXME: inserir anexo

% PPT: ooad-final-researchpaper, slide 12
% parece ser baseado no artigo 2004_Huo,M
\begin{itemize}
    \item \textbf{Estáticas}: não envolvem a execução de código, baseia-se na verificação de documentos, requisitos, do código-fonte. Técnicas estáticas tradicionais incluem revisão, inspeção, \emph{walk-through}, análise do algoritmo, etc.
    \item \textbf{Dinâmicas}: envolvem a execução do programa para verificar a existência de possíveis falhas e observar o comportamento e performance. Testes (unitários, integração, regressão, etc), verificações automáticas e simulações são exemplos de técnicas dinâmicas.
\end{itemize}

Apesar do teste ser crucial para as atividades de V\&V, \citeonline{PRESSMAN2010} lembra que as outras atividades de QA também são necessárias (revisões técnicas, auditorias de qualidade e configuração, monitoramento de performance, simulações, revisões de bancos de dados, dentre muitas outras). Outro fator importante: o teste não pode adicionar qualidade onde não existe, ela deve fazer parte de todo o fluxo de desenvolvimento.

% pag 213 - essentials of software engineering, tsui
% \citeonline{tsui2013} diz que todo software é composto por duas partes: uma estrutura estática (o código fonte) e um comportamento dinâmico (o software em execução). Alguns produtos intermediários do projeto (documentação de requisitos) só possuem uma estrutura estática.

Modelos de projeto formais utilizam métodos de análise estáticas e dinâmicas, enquanto os métodos ágeis focam mais em análise dinâmicas. Veremos a seguir a visão de qualidade e testes nos modelos formais e nos processos ágeis. % FIXME: é bom ter referências sobre as afirmações de agil

\section{QA nos Modelos de Projeto Formais}
\label{qa-modelos-formais}

% FIXME: os cinco estágios estão muito soltos e falta um início
\citeonline{WAZLAWICK2013} apresenta um modelo de maturidade organizacional em relação à qualidade, definido por Crosby (1979), baseado em cinco estágios:
\begin{enumerate}
    \item \emph{Desconhecimento:} quando a empresa sequer sabe que tem problemas com qualidade. Não há compreensão de que a qualidade seja um objetivo, ferramentas não são usadas ou conhecidas, e inspeções de qualidade não são realizadas.
    \item \emph{Despertar:} a empresa reconhece que tem problemas com a qualidade e que precisa começar a lidar com eles, mas ainda vê isso como um mal necessário, não como fonte de lucro.
    \item \emph{Alinhamento:} o gerenciamento de qualidade se torna uma ferramente institucional e os problemas vão sendo priorizados e resolvidos à medida que surgem.
    \item \emph{Sabedoria:} a prevenção de problemas, e não apenas sua correção, torna-se rotina na empresa. Problemas são identificados antes que surjam, e todos os processos e rotinas estão abertos a mudanças visando à melhoria da qualidade.
    \item \emph{Certeza:} a gestão da qualidade é uma constante e parte essencial do funcionamento da empresa. Quase todos os problemas são prevenidos e eliminados antes de surgirem.
\end{enumerate}

%Nos modelos formais de desenvolvimento (ex: cascata, espiral, incremental, etc)

% Para realizar a verificação da qualidade são utilizadas técnicas como \emph{revisões, inspeções} e \emph{testes sistemáticos}. Revisões e inspeções baseiam-se na revisão sistemática do produto por terceiros, para detectar defeitos, enquanto os testes sistemáticos são testes dinâmicos executados com o código produzido (ex.: testes unitários, testes de integração, testes de regressão, etc). \cite{WAZLAWICK2013}. % FIXME: inserir anexo ou nota de rodapé sobre onde ler mais detalhadamente sobre testes

\citeonline{crispin2009} aponta que nos modelos de projeto formais a equipe de testes trabalha de maneira totalmente independente dos desenvolvedores, e não acompanha o projeto desde o início. A equipe de QA não possui controle sobre o código era escrito, ou se os programadores testavam o código mas mesmo assim existe a expectativa de que consigam melhorar a qualidade do código após longos ciclos de desenvolvimento. Outro fator importante a ser levado em consideração é que os times de desenvolvimento não influenciam as particularidades de um recurso, ou o seu funcionamento. Cada um tende a se especializar em uma área do código enquanto a equipe de QA estuda os documentos de requisitos para escrever os testes e aguardar o código ser concluído e entregue para poder iniciar os testes.

\citeonline{SOMMERVILLE2011} reforça que o objetivo das atividades de QA não é avaliar a performance da equipe de desenvolvimento, mas detectar erros, que inevitavelmente ocorrerão. A \textbf{equipe responsável de QA} deve ser sensível às questões individuais dos desenvolvedores, e deve criar uma cultura de suporte à descoberta de erros sem implicação de culpa. Ele também cita alguns conceitos \textbf{errados} comuns envolvendo QA e testes:

\begin{itemize}
    \item desenvolvedores não precisam realizar testes;
    \item o software deve ser \emph{jogado} para estranhos o testem sem piedade; e
    \item que os testadores só devem se envolver com o projeto durante a fase de testes.
\end{itemize}

Os desenvolvedores devem se responsabilizar pela execução e corretude dos testes unitários, e podem até mesmo realizar os testes de integração, e somente após a arquitetura do software estar concluída que uma equipe independente de testes deve participar \cite{SOMMERVILLE2011}.

% ----------------------------------------------------------
% Seção
% ----------------------------------------------------------
\section{QA nos Modelos Ágeis}
% http://www.infoq.com/news/2014/08/quality-velocity-agile
%http://www.velocitypartners.net/blog/2014/05/06/read-my-lips-agile-isnt-fast/
\begin{citacao}
Ágil é uma ``Jogada de Qualidade'' e se você jogar com integridade, compromisso e equilíbrio ele pode se tornar uma ótima ``Jogada de Velocidade''. Porém a velocidade não vem de graça...\cite{galen2014}
\end{citacao}

Curioso notar que vários autores, seja em livro ou em artigos online, indicam que os Métodos Ágeis são mal compreendidos e o foco deles não são em velocidade e aumento de produtividade, mas em aumento de qualidade do projeto, e que caso seja bem aplicado e adotado, ele pode resultar em um aumento de velocidade, porém, caso contrário \citeonline{galen2015} afirma que o uso de um método ágil pode resultar em um processo mais lento do que um modelo formal.

% location 863 kindle - more agile testing
\citeonline{crispin2014} declaram que a pressão para atingir prazos irreais faz com que os desenvolvedores tomem decisões para entregar o produto e não para agradar ao cliente, sacrificando a qualidade final e gerando uma \emph{dívida técnica}. Eis a definição de \citeonline{fowler2003} para a dívida técnica:
% http://www.infoq.com/br/news/2009/10/dissecting-technical-debt
% http://www.infoq.com/br/news/2010/12/how-to-pay-down-technical-debt

\begin{citacao}
A dívida técnica é similar à dívida financeira. Assim como a dívida financeira, a dívida técnica exige o pagamento de juros. Estes vem na forma de esforço extra, que devem ser pagos em desenvolvimentos futuros por conta da escolha de um design mais rápido e de baixa qualidade. Nós podemos optar por continuar pagando estes juros ou quitar de uma vez a dívida fazendo uma refatoração, transformando um design de baixa qualidade em um design melhor. Apesar dos custos para saldar a dívida, ganhamos reduzindo os juros no futuro.
\end{citacao}

Ou seja, tempo é um fator extremamente importante, e a equipe precisa de tempo para pensar, para aprender nova técnicas e para realizar pequenos experimentos que podem ser úteis na resolução de problemas \cite{crispin2014}.

Utilizando os cinco estágios maturidade de Crosby, apresentado na seção \autoref{qa-modelos-formais}, podemos interpretar que os projetos ágeis já iniciam no terceiro estágio, alinhamento, tendo em vista a preocupação constante com os testes e o gerenciamento da qualidade de cada iteração. Mas a simples existência dos testes não garante a qualidade, é necessário que haja consciência da importância de cada etapa do processo e do objetivo a ser alcançado em cada iteração. \citeonline{galen2015} lista alguns problema encontrado por falta de conhecimento ou conhecimento superficial das técnicas: uma empresa adotava a criação de testes automatizados e integração contínua, porém não sabiam criar \emph{user stories} e não sabiam indicar quando uma tarefa estava concluída (\emph{Definition of Done - DoD}). % location 280 kindle - three pillars

% location 834 kindle - more agile testing
\citeonline{crispin2014} indica que métodos ágeis já são adotados por grande empresas, equipes distribuídas e que elas estão encontrando maneiras de gerar Produtos Minimamente Viáveis (\emph{minimum viable products - MVPs}), através de uso de entregas iterativas e rápidos processos de feedback e aprendizado. Um caso atual é o da empresa Spotify, que cresceu entre 2006 à 2014 para mais 1500 funcionários em 30 países, onde inicialmente adotavam Scrum e passaram a adotar um método ágil próprio pois as regras do Scrum foram consideradas limitantes, então eles passaram a testar diversas técnicas e a adotar/adaptar as que funcionaram. Um exemplo disso é eles não possuírem equipes de testes, pois não funcionou na cultura de desenvolvimento da empresa, então cada equipe possui um membro que cuida dos testes, mas essa função não é exclusiva \cite{spotify2014}. % FIXME: confirmar essas informações sobre a equipe de testes

% anotação minha
Curioso notar como ainda persiste a separação entre equipe de teste e de desenvolvimento, enquanto a visão difundida, na visão do autor, sobre a aplicação de metodologias ágeis é a existência de uma pequena equipe única.

% location 744 kindle - agile testing
\citeonline{crispin2009} lembram que todos os membros de uma equipe ágil devem focar em entregar um produto de alta qualidade que provenha valor à um negócio, e a função dos testadores é garantir que o produto entregue tenha a qualidade que o cliente precisa. Todos devem se sentir responsáveis por entregar um produto com a melhor qualidade possível, e desfrutar de suas atividades. Lembrem do Manifesto Ágil\footnote{Manifesto Ágil: http://agilemanifesto.org/iso/ptbr/}, onde o foco nos indivíduos é maior que o foco no processo.% no livro dos pilares também tem algo sobre isso, é bom complementar

% location 792 kindle - agile testing
Os princípios ágeis encorajam os membros de uma equipe a participar de múltiplas atividades e de realizar qualquer tipo de tarefa. Muitos praticantes dos princípios ágeis desencorajam papéis especializados, no entanto cada equipe deve decidir de quais especialidades serão necessárias: arquitetos, desenvolvedores, especialistas de segurança, dentre outras (\autoref{fig:agile-development-team}). Testadores também fazem parte da equipe de desenvolvimento pois o teste é um componente central do desenvolvimento de software ágil e são eles que garantem que o produto atende às expectativas de qualidade do cliente \cite{crispin2009}.

\begin{figure}[H]
    \centering
    \caption{Equipe de desenvolvimento ágil}
    \graphicspath{ {./graphics/agile/} }
    \includegraphics[scale=1.0]{agile-development-team}
    \label{fig:agile-development-team}
    \legend{Fonte: \citeonline{rasmusson2010}}
\end{figure}

% location 812 kindle - agile testing
\citeonline{crispin2009} também afirma que em algumas equipes não existe nenhum membro que se identifique como ``testador'', porém ele julga necessário que exista alguém que interaja diretamente com o cliente para a criação das histórias das iterações, para garantir que os testes sejam automatizados e que os testes estejam passando. É um membro com tais conhecimentos é vital para o sucesso do projeto.

% ----------------------------------------------------------
% Seção
% ----------------------------------------------------------
\section{Teste tradicional VS teste ágil}
%location 894-901 kindle - agile testing

\begin{figure}[H]
    \centering
    \caption{Comparativo de fases de um método tradicional (cascata) e método ágil}
    \graphicspath{ {./graphics/agile/} }
    \includegraphics[scale=1.0]{waterfall-vs-agile}
    \label{fig:waterfall-vs-agile}
    \legend{Fonte: \citeonline{crispin2009}}
\end{figure}

In the phased approach diagram, it is clear that testing happens at the end, right before release. The diagram is idealistic, because it gives the impression there is as much time for testing as there is for coding. In many projects, this is not the case. The testing gets “squished” because coding takes longer than expected, and because teams get into a code-and-fix cycle at the end.

Agile is iterative and incremental. This means that the testers test each increment of coding as soon as it is finished. An iteration might be as short as one week, or as long as a month. The team builds and tests a little bit of code, making sure it works correctly, and then moves on to next piece that needs to be built. Programmers never get ahead of the testers, because a story is not “done” until it has been tested. We’ll talk much more about this throughout the book.

Every project, every team, and sometimes every iteration is different. How your team solves problems should depend on the problem, the people, and the tools you have available. As an agile team member, you will need to be adaptive to the team’s needs.

Rather than creating tests from a requirements document that was created by business analysts before anyone ever thought of writing a line of code, someone will need to write tests that illustrate the requirements for each story days or hours before coding begins. This is often a collaborative effort between a business or domain expert and a tester, analyst, or some other development team member. Detailed functional test cases, ideally based on examples provided by business experts, flesh out the requirements. Testers will conduct manual exploratory testing to find important bugs that defined test cases might miss. Testers might pair with other developers to automate and execute test cases as coding on each story proceeds. Automated functional tests are added to the regression test suite. When tests demonstrating minimum functionality are complete, the team can consider the story finished.

Whatever flavor of development you’re using, pretty much the same elements of a software development life cycle need to happen. The difference with agile is that time frames are greatly shortened, and activities happen concurrently. Participants, tests, and tools need to be adaptive.

The most critical difference for testers in an agile project is the quick feedback from testing. It drives the project forward, and there are no gatekeepers ready to block project progress if certain milestones aren’t met.

While some teams do seem to use the “agile” buzzword to justify simply doing whatever they want, true agile teams are all about repeatable quality as well as efficiency. In our experience, an agile team is a wonderful place to be a tester.

% ----------------------------------------------------------
% Seção
% ----------------------------------------------------------
\section{Transição de método formal para método ágil}
%location 811 kindle - agile testing

O processo de transição de equipes tradicionais para adotar métodos ágeis normalmente é um choque, principalmente quanto à ideia de gerar código funcional dentro de uma ou duas semanas. Dúvidas comuns são: como programar e testar tão rapidamente? Como ter tempo para automatizar os testes? Como controlar código problemático que será entregue para produção?. \cite{crispin2009} citam dois casos de experiência própria que ilustram bem essa situação:

No primeiro caso é uma experiência própria de Lisa Crispin \cite{crispin2009}, sobre o seu primeiro contato com uma equipe ágil (que adotava \emph{eXtreme Programming - XP}), trabalhando como a única testadora em uma equipe de oito programadores, e ficou chocada ao ver que o produto da primeira iteração facilmente travava quando dois usuários logavam ao mesmo tempo. Seu \emph{coach}\footnote{\emph{Coach} é um papel do eXtreme Programming onde a pessoa auxilia o time a aprender e se manter no processo. Fonte: \url{ http://epf.eclipse.org/wikis/xp/xp/roles/xp_coach_60023190.html}} lhe explicou que naquele momento a necessidade do cliente era exibir o sistema à potenciais compradores, e confiabilidade e robustez não eram o foco. Nesse momento ela percebeu que quem determina os critérios de qualidade são as necessidades do cliente a cada iteração, e sua função seria escrever os testes que garantissem essa qualidade.

O segundo caso é uma experiência curiosa é contada por Jonathan Rasmusson \cite{crispin2009}, que trabalhava em uma equipe que adotava XP e um novo membro foi adicionado: uma testadora. Ele não viu sentido nessa contratação, pois por adotarem XP os desenvolvedores escreviam todos os testes, porém dentro de seis meses ele percebeu a importância. Por não ter que focar nos testes de baixo nível, que estavam automatizados, ela pode focar em usabilidade, realizar testes exploratórios ao pensar de maneiras que os desenvolvedores não haviam antecipado, trabalhar em conjunto com o cliente para escrever casos de testes para as próximas funcionalidades, e participar de desenvolvimento em par com os programadores. Rasmusson reforça que testadores ágeis não ficam aguardando trabalho, eles procuram maneiras de contribuir com o ciclo de desenvolvimento.

\subsection{Dez princípios para testadores ágeis}
% location 973 kindle - agile testing
% location 1239 kindle - more agile testing

% FIXME: por um texto introdutório

\begin{enumerate}
    \item Prover \emph{feedback} contínuo;
    \item Entregar algo de valor ao cliente;
    \item Possibilitar comunicação face a face;
    \item Ter coragem;
    \item Manter tudo simples;
    \item Praticar melhoria contínua;
    \item Responder à mudanças;
    \item Auto-organizar;
    \item Focar nas pessoas;
    \item Divertir-se.
\end{enumerate}

\subsection{Logística da Equipe}
% localion 1726 kindle - agile testing
Algumas empresas acreditam ser importante que a equipe de QA seja independente para ter uma visão objetiva sobre a qualidade do produto. 

Eis alguns motivos para a separação:

\begin{itemize}
    \item It is important to have that independent check and audit role.
    \item The team can provide an unbiased and outside view relating to the quality of the product.
    \item If testers work too closely with developers, they will start to think like developers and lose their customer viewpoint.
    \item If the testers and developers report to the same person, there is a danger that the priority becomes delivering any code rather than delivering tested code.
\end{itemize}

% continuar leitura sobre este tópico

% location 451 kindle - agile samurai
Agile methods like Scrum and XP don’t have a lot of formal roles when it comes to projects. There are people who know what needs to be built (customers) and people who can build it (the development team).


\begin{figure}[H]
    \centering
    \caption{Equipe tradicional e equipe ágil}
    \graphicspath{ {./graphics/agile/} }
    \includegraphics[scale=1.0]{functional-and-agile-teams}
    \label{fig:functional-and-agile-teams}
    \legend{Fonte: \citeonline{crispin2009}}
\end{figure}

\subsection{Abordagem time coeso (\emph{whole-team approach})}
% isso tambem é explorado no XP: http://pt.wikipedia.org/wiki/Programa%C3%A7%C3%A3o_extrema
% location 934 kindle - agile testing

One of the biggest differences in agile development versus traditional development is the agile “whole-team” approach. With agile, it’s not only the testers or a quality assurance team who feel responsible for quality. We don’t think of “departments,” we just think of the skills and resources we need to deliver the best possible product. The focus of agile development is producing high-quality software in a time frame that maximizes its value to the business. This is the job of the whole team, not just testers or designated quality assurance professionals. Everyone on an agile team gets “test-infected.” Tests, from the unit level on up, drive the coding, help the team learn how the application should work, and let us know when we’re “done” with a task or story.

An agile team must possess all the skills needed to produce quality code that delivers the features required by the organization. While this might mean including specialists on the team, such as expert testers, it doesn’t limit particular tasks to particular team members. Any task might be completed by any team member, or a pair of team members. This means that the team takes responsibility for all kinds of testing tasks, such as automating tests and manual exploratory testing. It also means that the whole team thinks constantly about designing code for testability.

The whole-team approach involves constant collaboration. Testers collaborate with programmers, the customer team, and other team specialists— and not just for testing tasks, but other tasks related to testing, such as building infrastructure and designing for testability.

The whole-team approach means everyone takes responsibility for testing tasks. It means team members have a range of skill sets and experience to employ in attacking challenges such as designing for testability by turning examples into tests and into code to make those tests pass. These diverse viewpoints can only mean better tests and test coverage.

Most importantly, on an agile team, anyone can ask for and receive help. The team commits to providing the highest possible business value as a team, and the team does whatever is needed to deliver it. Some folks who are new to agile perceive it as all about speed. The fact is, it’s all about quality— and if it’s not, we question whether it’s really an “agile” team.

\subsubsection{Risco da falta de coesão}
% trecho interessante: risco de falta de coesão e importância do tester
% location 1224 kindle - agile testing
If you’re a tester, and you’re not invited to attend planning sessions, stand-ups, or design meetings, you might be in a situation where testers are viewed as somehow apart from the development team.

If you are invited to these meetings but you’re not speaking up, then you’re probably creating a perception that you aren’t really part of the team. If business experts are writing stories and defining requirements all by themselves, you aren’t participating as a tester who’s a member of an agile team.

If this is your situation, your team is at risk. Hidden assumptions are likely to go undetected until late in the release cycle. Ripple effects of a story on other parts of the system aren’t identified until it’s too late. The team isn’t making the best use of every team member’s skills, so it’s not going to be able to produce the best possible software. Communication might break down, and it’ll be hard to keep up with what the programmers and customers are doing. The team risks being divided in an unhealthy way between developers and testers, and there’s more potential that the development team will become isolated from the customer team.

How can you avoid this peril? See if you can arrange to be located near the developers. If you can’t, at least come to their area to talk and pair test. Ask them to show you what they’re working on. Ask them to look at the test cases you’ve written. Invite yourself to meetings if nobody else has invited you. Make yourself useful by testing and providing feedback, and become a necessity to the team.

Help customers develop their stories and acceptance tests. Push the “whole team” attitude, and ask the team to work on testing problems. If your team is having trouble adapting to agile development, suggest experimenting with some new ideas for an iteration or two. Propose adopting the “Power of Three” rule to promote good communication. Use the information in this book to show that testers can help agile teams succeed beyond their wildest expectations.

\subsection{Barreiras para adoção de times de teste/QA ágeis}
% esta seção me parece mais um guia para uma pessoa que vai implementar um método ágil
% location 1426 kindle - agile testing
\begin{itemize}
    \item perda de identidade: onde os testadores desejam manter independência da equipe de desenvolvimento;
    \item funções adicionais: um novo especialista pode ser necessário para sobrepor um obstáculo, portanto é necessário que todos saibam quais suas funções; % achei este item estranho
    \item falta de treinamento: testadores incorporados em equipes... % FIXME: continuar
    \item falta de compreensão de conceitos ágeis: nem todas as equipes ágeis são iguais. Existem diversas abordagens (XP, Scrum, Crystal Clear, Feature Driven Development - FDD, etc), abordagens customizadas, normalmente uma mistura das já citadas, ou até mesmo abordagens totalmente novas. O importante é que a equipe chegue a um consenso sobre como proceder e que trabalhe de forma coesa. Mas é importante notar que a simples adoção de iterações e ausência de documentação tradicional não resulta em uma abordagem ágil; % não sei se vou manter este trecho sobre mini-cascata - location 1458 kindle - agile testing
    \item experiências passadas negativas: as pessoas podem acabar desistindo de adotar o processo, caso não sejam treinadas, acompanhadas e que tenham tempo para se ajustar;
    \item diferenças culturais entre funções: cada pessoa está acostumada com um método de trabalho próprio, focado apenas em sua atividade.
\end{itemize}

% \begin{figure}[H]
%     \centering
%     \caption{Processo mini-cascata}
%     \graphicspath{ {./graphics/agile/} }
%     \includegraphics[scale=1.0]{mini-waterfall}
%     \label{fig:mini-waterfall}
%     \legend{Fonte: \citeonline{crispin2009}}
% \end{figure}

\subsubsection{Hipótese da qualidade negociável}
\citeonline{fowler2011} relata que uma situação bastante comum é a crença em algo chamado \emph{hipótese da qualidade negociável}, que diz que é possível realizar uma trocar entre recursos: ``é melhor para o produto adicionar um recurso visível para o usuário (qualidade externa) do que refatorar código (qualidade interna)''.

Se ela lógica for considerada válida, então quando valerá a pena investir recurso na qualidade interna? O objetivo, muitas vezes não compreendido, da qualidade interna é acelerar a velocidade do projeto ao evitar que código ruim e problemático seja gerado, evitando perdas futuras de tempo, facilitando sua compreensão, etc. Portanto, deve-se primeiro analisar se uma refatoração trará estes benefícios, e apenas no caso dele não ser vantajoso que ele deve ser evitado.


%%%%%%

\subsection{Quadrantes de teste ágil}
\begin{figure}[H]
    \centering
    \caption{Quatro quadrantes de teste ágil}
    \graphicspath{ {./graphics/agile/} }
    \includegraphics[scale=0.6]{Quadrante-Teste-Agil}
    \label{fig:quatro-quadrantes}
    \legend{Fonte: \citeonline{nogueira2013}}
\end{figure}

blablabla

% location 804 kindle - more agile testing
% FIXME: mover este trecho para um local mais adequado
Power of Three - Three Amigos, George Dinwiddie: onde um cliente, programador e testador colaboram quando existir um questionamento sobre como uma funcionalidade deve agir.

\subsection{Pilares da qualidade ágil}
Proposta de modelo criado por \citeonline{galen2015} para auxiliar na compreensão e aplicação de testes e obtenção de um software de qualidade. A intenção é ser um guia e não um modelo rígido e nem uma visão definitiva sobre teste ágil, existindo outras propostas como os Quatro Quadrantes do Teste Ágil (\emph{4 Quadrants of Agile Testing}) de \citeonline{crispin2009}.

\begin{figure}[H]
    \centering
    \caption{Três Pilares do Teste e Qualidade Ágil}
    \graphicspath{ {./graphics/agile/} }
    \includegraphics[scale=0.6]{three-pillars}
    \label{fig:three-pillars}
    \legend{Fonte: \citeonline{crispin2009}}
\end{figure}


\subsubsection{Automação e Ferramentas de Teste}
% location 316-318 kindle - three pillars
This pillar is the technology-side of quality and testing and is not simply focused towards testing and testers. It includes tooling, execution of the Automation Test Pyramid, Continuous Integration, deep use of Extreme Programming technical practices, and support for ALM distributed collaboration tools.

Normalmente é o que mais atrai as empresas/desenvolvedores, devido à atratividade do uso de ferramentas e da tecnologia para a resolução dos problemas.

% location 492 kindle - three pillars
Pillar 1 is the technically focused pillar and the one that agile organizations are usually quite comfortable discussing – at least at a high level. It's where automation, in general, is located. Terms like Continuous Integration, Continuous Deployment, Cucumber, TDD, ATDD, BDD5, virtualization, and DevOps are actively discussed and implemented here.

% location 566 kindle - three pillars
The first change is to take a whole-team view. Instead of the testers being responsible for testing and writing all of the test automation, it becomes a whole-team responsibility. The developers take most of the ownership for unit-level automation at the base of the pyramid, but testers can operate there as well.

For example, as of this writing, these are common tools leveraged at each of the three tiers:
\begin{enumerate}
    \item UI tier: Selenium, Watir, or traditional UFT
    \item Middle tier: FitNesse, Robot Framework, JBehave, and Cucumber
    \item Unit tier: xUnit family variants for example JUnit or NUnit for Java and. Net respectively
\end{enumerate}

The other consideration is that there are extensions to several of these. For example, both Robot Framework and Cucumber have Selenium plug-ins so that they can drive the UI, as well as, the middle tier. This implies that the middle tier tooling, and automated tests for that matter, can extend or blend into the lower and upper tiers.

% importancia do trabalho em par
I've found it surprising, but true, that many developers don't really understand how to write effective unit tests – especially when you expect those tests to be part of a cohesive automation whole. Partnering with testers can help immensely when designing unit tests, even if the testers can't program in the language du jour. I've also found that training can really be a force multiplier here.

And a final caution – don't try to automate everything! The strategy or business case for 100\% automation is nonsense. I'm usually against setting any sort of "magic number" when establishing goals for automation levels of coverage. Rather, ask the team to look at each user story and determine the appropriate level of unit, middle-tier, and UI tests in order to adequately cover it compared to its complexity and value.

\begin{figure}[H]
    \centering
    \caption{Pirâmide de teste automatizado tradicional}
    \graphicspath{ {./graphics/agile/} }
    \includegraphics[scale=1.0]{testing-pyramid-traditional}
    \label{fig:testing-pyramid-traditional}
    \legend{Fonte: \citeonline{galen2015}}
\end{figure}

\begin{figure}[H]
    \centering
    \caption{Pirâmide de teste automatizado ágil}
    \graphicspath{ {./graphics/agile/} }
    \includegraphics[scale=1.0]{testing-pyramid-agile}
    \label{fig:testing-pyramid-agile}
    \legend{Fonte: \citeonline{galen2015}}
\end{figure}

\begin{figure}[H]
    \centering
    \caption{Software de automatização iContact (2012)}
    \graphicspath{ {./graphics/agile/} }
    \includegraphics[scale=1.0]{icontact-automation-pyramid}
    \label{fig:icontact-automation-pyramid}
    \legend{Fonte: \citeonline{galen2015}}
\end{figure}

\subsubsection{Teste de Software}
% location 327-330 kindle - three pillars
This pillar is focused towards the profession of testing. Towards solid testing practices, not simply agile testing practices, but leveraging the teams' past testing experience, skills, techniques, and tools. This is the place where agile teams move from a trivial view to software testing, which only looks at TDD, ATDD, and developer-based testing, towards a more holistic view of agile quality and testing.

% location 833 kindle - three pillars
Many agile teams consider agile testing to only include:
\begin{itemize}
    \item Unit-level testing; including perhaps automating the unit tests;
    \item Functional level testing at a User Story level;
    \item Story Acceptance-level testing; and
    \item On-the-fly Continuous Integration leveraging common CI tooling.
\end{itemize}

Testing often stops after these steps are accomplished within each team and sprint. Now, in certain contexts, for example, small web based products that are relatively new, this may well be all that is needed. But in many, many more contexts, there is a lot more testing activity that is required. Repetitive testing of previously delivered functionality is also a consideration that is often lost in the frenzy of delivering "done" software within each sprint.

\subsubsection{Práticas da Equipe}
% location 337-340 kindle - three pillars
Finally, this pillar is focused towards cross-team collaboration, team-based standards, quality attitudes, agile mindsets, and, most importantly, towards building things right. Consider this the soft skills area of the Three Pillars, where direction is provided for how each team will operate. You could also consider them "rules of engagement".

% ---
% Finaliza a parte no bookmark do PDF, para que se inicie o bookmark na raiz
% ---
\bookmarksetup{startatroot}% 
% ---

% ==========================================================
% CAPITULO
% ==========================================================
\chapter*[Conclusão]{Conclusão}
\addcontentsline{toc}{chapter}{Conclusão}

\emph{*em desenvolvimento*}

%\lipsum[31-33]

% ----------------------------------------------------------
% ELEMENTOS PÓS-TEXTUAIS
% ----------------------------------------------------------
\postextual


% ----------------------------------------------------------
% Referências bibliográficas
% ----------------------------------------------------------
%\bibliography{abntex2-modelo-references}
\bibliography{andrei-bibtex}

% ----------------------------------------------------------
% Glossário
% ----------------------------------------------------------
%
% Consulte o manual da classe abntex2 para orientações sobre o glossário.
%
%\glossary

%---------------------------------------------------------------------
% INDICE REMISSIVO
%---------------------------------------------------------------------

\printindex

\end{document}
