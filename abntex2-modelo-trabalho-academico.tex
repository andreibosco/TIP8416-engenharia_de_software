%% abtex2-modelo-trabalho-academico.tex, v-1.7.1 laurocesar
%% Copyright 2012-2013 by abnTeX2 group at http://abntex2.googlecode.com/ 
%%
%% This work may be distributed and/or modified under the
%% conditions of the LaTeX Project Public License, either version 1.3
%% of this license or (at your option) any later version.
%% The latest version of this license is in
%%   http://www.latex-project.org/lppl.txt
%% and version 1.3 or later is part of all distributions of LaTeX
%% version 2005/12/01 or later.
%%
%% This work has the LPPL maintenance status `maintained'.
%% 
%% The Current Maintainer of this work is the abnTeX2 team, led
%% by Lauro César Araujo. Further information are available on 
%% http://abntex2.googlecode.com/
%%
%% This work consists of the files abntex2-modelo-trabalho-academico.tex,
%% abntex2-modelo-include-comandos and abntex2-modelo-references.bib
%%

% ------------------------------------------------------------------------
% ------------------------------------------------------------------------
% abnTeX2: Modelo de Trabalho Academico (tese de doutorado, dissertacao de
% mestrado e trabalhos monograficos em geral) em conformidade com 
% ABNT NBR 14724:2011: Informacao e documentacao - Trabalhos academicos -
% Apresentacao
% ------------------------------------------------------------------------
% ------------------------------------------------------------------------

\documentclass[
	% -- opções da classe memoir --
	12pt,				% tamanho da fonte
	openright,			% capítulos começam em pág ímpar (insere página vazia caso preciso)
	twoside,			% para impressão em verso e anverso. Oposto a oneside
	a4paper,			% tamanho do papel. 
	% -- opções da classe abntex2 --
	%chapter=TITLE,		% títulos de capítulos convertidos em letras maiúsculas
	%section=TITLE,		% títulos de seções convertidos em letras maiúsculas
	%subsection=TITLE,	% títulos de subseções convertidos em letras maiúsculas
	%subsubsection=TITLE,% títulos de subsubseções convertidos em letras maiúsculas
	% -- opções do pacote babel --
	english,			% idioma adicional para hifenização
	brazil,				% o último idioma é o principal do documento
	]{abntex2}


% ---
% PACOTES
% ---

% ---
% Pacotes fundamentais 
% ---
\usepackage{cmap}				% Mapear caracteres especiais no PDF
\usepackage{lmodern}			% Usa a fonte Latin Modern			
\usepackage[T1]{fontenc}		% Selecao de codigos de fonte.
\usepackage[utf8]{inputenc}		% Codificacao do documento (conversão automática dos acentos)
\usepackage{lastpage}			% Usado pela Ficha catalográfica
\usepackage{indentfirst}		% Indenta o primeiro parágrafo de cada seção.
\usepackage{color}				% Controle das cores
\usepackage{graphicx}			% Inclusão de gráficos
\usepackage{abntex2-ufc}
% ---
		
% ---
% Pacotes adicionais, usados apenas no âmbito do Modelo Canônico do abnteX2
% ---
\usepackage{lipsum}				% para geração de dummy text
% ---

% ---
% Pacotes de citações
% ---
\usepackage[brazilian,hyperpageref]{backref}	 % Paginas com as citações na bibl
\usepackage[alf]{abntex2cite}	% Citações padrão ABNT

% --- 
% CONFIGURAÇÕES DE PACOTES
% --- 

% ---
% Configurações do pacote backref
% Usado sem a opção hyperpageref de backref
\renewcommand{\backrefpagesname}{Citado na(s) página(s):~}
% Texto padrão antes do número das páginas
\renewcommand{\backref}{}
% Define os textos da citação
\renewcommand*{\backrefalt}[4]{
	\ifcase #1 %
		Nenhuma citação no texto.%
	\or
		Citado na página #2.%
	\else
		Citado #1 vezes nas páginas #2.%
	\fi}%
% ---


% ---
% Informações de dados para CAPA e FOLHA DE ROSTO
% ---
\titulo{AT2 - Mecanismos de implementação da qualidade de software com perspectivas sobre os testes, inspeções e métodos ágeis}
\autor{Andrei Bosco Bezerra Torres}
\local{Fortaleza - CE}
\data{Abril de 2015}
\orientador[Orientadores]{Professor: José Marques Soares}
\instituicao{%
  Universidade Federal do Ceará -- UFC
  \par
  Departamento de Engenharia de Teleinformática
  \par
  Programa de Pós-Graduação em Engenharia de Teleinformática}
\tipotrabalho{Atividade de mestrado}
% O preambulo deve conter o tipo do trabalho, o objetivo, 
% o nome da instituição e a área de concentração 
\preambulo{Trabalho da disciplina de Engenharia de Software apresentado como parte das atividades necessárias.}
% ---


% ---
% Configurações de aparência do PDF final

% alterando o aspecto da cor azul
\definecolor{blue}{RGB}{41,5,195}

% informações do PDF
\makeatletter
\hypersetup{
     	%pagebackref=true,
		pdftitle={\@title}, 
		pdfauthor={\@author},
    	pdfsubject={\imprimirpreambulo},
	    pdfcreator={LaTeX with abnTeX2},
		pdfkeywords={abnt}{latex}{abntex}{abntex2}{trabalho acadêmico}, 
		colorlinks=true,       		% false: boxed links; true: colored links
    	linkcolor=blue,          	% color of internal links
    	citecolor=blue,        		% color of links to bibliography
    	filecolor=magenta,      		% color of file links
		urlcolor=blue,
		bookmarksdepth=4
}
\makeatother
% --- 

% --- 
% Espaçamentos entre linhas e parágrafos 
% --- 

% O tamanho do parágrafo é dado por:
\setlength{\parindent}{1.3cm}

% Controle do espaçamento entre um parágrafo e outro:
\setlength{\parskip}{0.2cm}  % tente também \onelineskip

% ---
% compila o indice
% ---
\makeindex
% ---

% ----
% Início do documento
% ----
\begin{document}

% Retira espaço extra obsoleto entre as frases.
\frenchspacing 

% ----------------------------------------------------------
% ELEMENTOS PRÉ-TEXTUAIS
% ----------------------------------------------------------
% \pretextual

% ---
% Capa
% ---
\imprimircapa
% ---

% ---
% Folha de rosto
% (o * indica que haverá a ficha bibliográfica)
% ---
%\imprimirfolhaderosto*
\imprimirfolhaderosto
% ---

% ---
% RESUMOS
% ---

% resumo em português
\begin{resumo}
 O foco deste trabalho é abordar sobre qualidade de software
 e todo o processo envolvido no desenvolvimento e entrega de
 um software de alto padrão de qualidade, desde as etapas de
 verificação e validação através de testes e inspeções, até 
 os modelos de qualidade, métricas e gestão. Também será
 abordado a aplicação de modelos de qualidade em projetos 
 que adotam métodos ágeis e como isso pode impactar o resultado
 final do produto.

 \vspace{\onelineskip}
    
 \noindent
 \textbf{Palavras-chaves}: testes. validação. verificação.
 qualidade de software. métodos ágeis.
\end{resumo}

% ---
% inserir lista de ilustrações
% ---
\pdfbookmark[0]{\listfigurename}{lof}
\listoffigures*
\cleardoublepage
% ---

% ---
% inserir lista de tabelas
% ---
\pdfbookmark[0]{\listtablename}{lot}
\listoftables*
\cleardoublepage
% ---

% ---
% inserir lista de abreviaturas e siglas
% ---
\begin{siglas}
  \item[Fig.] Area of the $i^{th}$ component
  \item[456] Isto é um número
  \item[123] Isto é outro número
  \item[IEEE] \emph{Institute of Electrical and Electronics Engineers} - Instituto de Engenheiros Eletricistas e Eletrônicos
  \item[ISO] \emph{International Organization for Standardization} - Organização Internacional para Padronização
  \item[QA] \emph{Quality Assurance} - Garantia de Qualidade
\end{siglas}
% ---

% ---
% inserir lista de símbolos
% ---
% \begin{simbolos}
%   \item[$ \Gamma $] Letra grega Gama
%   \item[$ \Lambda $] Lambda
%   \item[$ \zeta $] Letra grega minúscula zeta
%   \item[$ \in $] Pertence
% \end{simbolos}
% ---

% ---
% inserir o sumario
% ---
\pdfbookmark[0]{\contentsname}{toc}
\tableofcontents*
\cleardoublepage
% ---



% ----------------------------------------------------------
% ELEMENTOS TEXTUAIS
% ----------------------------------------------------------
\textual

% ----------------------------------------------------------
% Introdução
% ----------------------------------------------------------
\chapter*[Introdução]{Introdução}
\addcontentsline{toc}{chapter}{Introdução}

Este documento e seu código-fonte são exemplos de referência de uso da classe
\textsf{abntex2} e do pacote \textsf{abntex2cite}. O documento 
exemplifica a elaboração de trabalho acadêmico (tese, dissertação e outros do
gênero) produzido conforme a ABNT NBR 14724:2011 \emph{Informação e documentação
- Trabalhos acadêmicos - Apresentação}.

A expressão ``Modelo Canônico'' é utilizada para indicar que \abnTeX\ não é
modelo específico de nenhuma universidade ou instituição, mas que implementa tão
somente os requisitos das normas da ABNT. Uma lista completa das normas
observadas pelo \abnTeX\ é apresentada em %\citeonline{abntex2classe}.

Sinta-se convidado a participar do projeto \abnTeX! Acesse o site do projeto em
\url{http://abntex2.googlecode.com/}. Também fique livre para conhecer,
estudar, alterar e redistribuir o trabalho do \abnTeX, desde que os arquivos
modificados tenham seus nomes alterados e que os créditos sejam dados aos
autores originais, nos termos da ``The \LaTeX\ Project Public
License''\footnote{\url{http://www.latex-project.org/lppl.txt}}.

Encorajamos que sejam realizadas customizações específicas deste exemplo para
universidades e outras instituições --- como capas, folha de aprovação, etc.
Porém, recomendamos que ao invés de se alterar diretamente os arquivos do
\abnTeX, distribua-se arquivos com as respectivas customizações.
Isso permite que futuras versões do \abnTeX~não se tornem automaticamente
incompatíveis com as customizações promovidas. Consulte
%\citeonline{abntex2-wiki-como-customizar} par mais informações.

Este documento deve ser utilizado como complemento dos manuais do \abnTeX\ 
%\cite{abntex2classe,abntex2cite,abntex2cite-alf} e da classe \textsf{memoir}
%\cite{memoir}. 

Esperamos, sinceramente, que o \abnTeX\ aprimore a qualidade do trabalho que
você produzirá, de modo que o principal esforço seja concentrado no principal:
na contribuição científica.

Equipe \abnTeX 

Lauro César Araujo


% ----------------------------------------------------------
% PARTE
% ----------------------------------------------------------
\part{Teoria}

% ----------------------------------------------------------
% Capitulo
% ----------------------------------------------------------
\chapter{Qualidade de Software}

% ----------------------------------------------------------
% Seção
% ----------------------------------------------------------
\section{Definição}

Definir \emph{qualidade} não é tão simples como parece. Você sabe identificar quando um produto é de qualidade mas normalmente não é fácil indicar o que lhe dá essa característica. No caso de qualidade para Engenharia de Software vários autores trazem suas definições e existem vários modelos e organizações que almejam criar uma estrutura para a criação de softwares de qualidade.

\begin{description}
    \item[Definição da \citeonline{IEEE1990} para Qualidade:] \hfill \\
        \begin{itemize}
            \item O grau em que um sistema, componente ou processo alcança os requerimentos especificados.
            \item O grau em que um sistema, componente ou processo atinge as necessidades ou expectativas de um cliente ou usuário.
        \end{itemize}
    \item[Definição da \citeonline{IEEE1990} para Garantia de Qualidade (QA):] \hfill \\
        \begin{itemize}
            \item Um padrão planejado e sistemático de todas as ações necessárias para garantir uma confiança adequada de que um item ou produto está em conformidade com os requisitos técnicos estabelecidos.
            \item Um conjunto de atividades destinadas a avaliar o processo pelo qual os produtos são desenvolvidos ou produzidos.
        \end{itemize}
    \item[Definição de \citeonline{PRESSMAN2010} para Qualidade de Software:] \hfill \\
        \begin{itemize}
            \item Um processo de software eficaz aplicada de uma maneira que cria um produto útil que oferece valor mensurável para quem produz e para quem usá-lo.
        \end{itemize}
\end{description}

Mas por que é tão difícil chegar a uma definição objetiva? Basicamente porque é quase impossível analisar objetivamente se um requisito foi cumprido adequadamente. Pode haver divergências de interpretação de um requisito entre o desenvolvedor e o cliente, ou pode haver conflito de expectativas entre diferentes \emph{stakeholders}. Devido à esses questionamentos, cabe à equipe de gerenciamento de qualidade julgar se um nível de qualidade foi alcançado. \citeonline{SOMMERVILLE2011} exemplifica algumas questões que auxiliam a validar a qualidade de um sistema:
\begin{enumerate}
  \item Os padrões de programação e documentação foram seguidos durante o processo de desenvolvimento?
  \item O software foi testado adequadamente?
  \item O software é suficientemente confiável para ser posto em produção?
  \item A performance do software é aceitável para uso normal?
  \item O software é utilizável?
  \item O software é bem estruturado e compreensível?
\end{enumerate}

\subsection{Fatores de Qualidade}

%(FIXME: página 673 de Software Engineering (9th Edition) - rever este parágrafo)
Note que a qualidade de um software não está sujeita apenas a analisar se uma funcionalidade foi implementada corretamente, mas também depende dos requisitos não-funcionais. \citeonline{BOEHM1978} sugere 15 atributos impostantes para a qualidade de software (ver Tabela~\ref{tab:atributos_qualidade} na página~\pageref{tab:atributos_qualidade}). Para \citeonline{SOMMERVILLE2011} estes atributos estão relacionados às seguintes características: dependability, usability, efficiency e maintainability, e lembra que é impossível um sistema ser otimizado em todos estes pontos, portanto é necessário ser analisado e definido quais atributos são vitais para a aplicação. Caso a eficiência seja crítica então as outras características podem ser sacrificadas.

% dicas de posicionamento de floats:
% http://www.andy-roberts.net/writing/latex/floats_figures_captions
\begin{table}[h]
    \centering
    \caption{Atributos de qualidade de software}
    \label{tab:atributos_qualidade}
    \begin{tabular}{c|c|c}
        Safety & Understandability & Portability \\ \hline
        Security & Testability & Usability \\ \hline
        Reliability & Adaptability & Reusability \\ \hline
        Resilience & Modularity & Efficiency \\ \hline
        Robustness & Complexity & Learnability
    \end{tabular}
    \legend{Fonte: \citeonline{BOEHM1978}}
\end{table}

% FIXME: rever e complementar este trecho
% ver página 430 de Software Engineering A Practitioner's Approach 7th Edition - Roger Pressman

Na visão de \citeonline{GARVIN1987} a qualidade deve ser analisada de um ponto de vista multidimensional, desde a conformância com os requisitos até o caráter estético da aplicação (ver Tabela \ref{tab:qualidade_garvin}). 

\begin{table}[h]
    \caption{Dimensões de qualidade de \citeonline{GARVIN1987}}
    \label{tab:qualidade_garvin}
    \begin{tabular}{p{3.7cm}|p{11cm}}
        \textbf{Performance} & o software entrega todo o conteúdo/funções/característica especificadas nos requerimentos e de maneira que agregue valor ao usuário? \\ \hline
        \textbf{Característica} & o software possui alguma característica que surpreenderá e agradará usuários finais novatos? \\ \hline
        \textbf{Confiabilidade} & o software disponibiliza suas funções sem falhas? \\ \hline
        \textbf{Conformidade} & o software está em conformidade com padrões locais e externos que são relevantes à aplicação? Ele segue convenções de código e de design? \\ \hline
        \textbf{Durabilidade} & O software pode ser mantido (alterado) e corrigido (debugado) sem efeitos colaterais indesejáveis? \\ \hline
        \textbf{Manutenibilidade} & o software pode ser mantido (alterado) e corrigido (debugado) em um espaço de tempo aceitável? \\ \hline
        \textbf{Estética} & o valor estético, elegância, a \emph{presença} do software.\\ \hline
        \textbf{Percepção} & Preconceitos (positivos ou negativos) que influenciam a percepção de qualidade. Um software anteriormente produzido por um desenvolvedor que produziu algo de baixa qualidade causará uma influência negativa na nova versão.\\
  \end{tabular}
  \legend{Fonte: \citeonline{PRESSMAN2010}}
\end{table}

Pode-se notar que qualidade é um conceito extremamente amplo e abrangente, e nem sempre simples de ser avaliado objetivamente. Com isso  %FIXME: parei aqui

% ----------------------------------------------------------
% Seção
% ----------------------------------------------------------
%\section{Importância}

% FIXME: decidir se mantenho esta seção
%E vale a pena investir recursos em busca de um conceito tão difícil de se definir?

% ----------------------------------------------------------
% Seção
% ----------------------------------------------------------
\section{Tipos de Qualidade}


\subsection{Qualidade de Projeto}


\subsection{Qualidade de Conformação}

% ----------------------------------------------------------
% Seção
% ----------------------------------------------------------
\section{Modelos de Qualidade}

\subsection{Modelo McCall}

% FIXME: tabela com conteúdo errado, apenas título certo
% ver página 431 de Software Engineering A Practitioner's Approach 7th Edition - Roger Pressman

\begin{table}[h]
    \caption{Fatores de qualidade de \citeonline{general1977factors}}
    \label{tab:qualidade_mccal}
    \begin{tabular}{p{3.7cm}|p{11cm}}
        \textbf{Performance} & o software entrega todo o conteúdo/funções/característica especificadas nos requerimentos e de maneira que agregue valor ao usuário? \\ \hline
        \textbf{Característica} & o software possui alguma característica que surpreenderá e agradará usuários finais novatos? \\ \hline
        \textbf{Confiabilidade} & o software disponibiliza suas funções sem falhas? \\ \hline
        \textbf{Conformidade} & o software está em conformidade com padrões locais e externos que são relevantes à aplicação? Ele segue convenções de código e de design? \\ \hline
        \textbf{Durabilidade} & O software pode ser mantido (alterado) e corrigido (debugado) sem efeitos colaterais indesejáveis? \\ \hline
        \textbf{Manutenibilidade} & o software pode ser mantido (alterado) e corrigido (debugado) em um espaço de tempo aceitável? \\ \hline
        \textbf{Estética} & o valor estético, elegância, a \emph{presença} do software.\\ \hline
        \textbf{Percepção} & Preconceitos (positivos ou negativos) que influenciam a percepção de qualidade. Um software anteriormente produzido por um desenvolvedor que produziu algo de baixa qualidade causará uma influência negativa na nova versão.\\
  \end{tabular}
  \legend{Fonte: \citeonline{PRESSMAN2010}}
\end{table}


\subsection{Modelo Boehm}


\subsection{Modelo FURPS (\emph{Functionality, Usability, Reliability, Performance,
Supportability})}


\subsection{Modelo Dromey}


\subsection{SquaRE - ISO/IEC 25010:2011 (\emph{Systems and software Quality Requirements
and Evaluation})}

% ----------------------------------------------------------
% Seção
% ----------------------------------------------------------
\section{Gestão da Qualidade}


\subsection{\emph{Walkthrough}}


\subsection{Inspeções Fagan}


\subsection{Método \emph{Cleanroom}}

% ----------------------------------------------------------
% Seção
% ----------------------------------------------------------
\section{Medição da Qualidade}

% ----------------------------------------------------------
% Seção
% ----------------------------------------------------------
\section{Padrões / Certificações}

% ----------------------------------------------------------
% Seção
% ----------------------------------------------------------
\section{Qualidade de Software e Métodos Ágeis}

% ----------------------------------------------------------
% Capitulo
% ----------------------------------------------------------

\chapter{Testes}

% ----------------------------------------------------------
% Seção
% ----------------------------------------------------------
\section{Definição}

\begin{citacao}
Por melhores que sejam as técnicas de modelagem e especificação de software, por mais disciplinada e experiente que seja a equipe de desenvolvimento, sempre haverá um fator que faz que o teste de software seja necessário: o erro humano. É um mito pensar que bons desenvolvedores, bem concentrados e com boas ferramentas serão capazes de desenvolver software sem erros (Beizer, 1990) **INSERIR NAS REFERÊNCIAS BIBLIOGRAFICAS.
\end{citacao}

Mesmo que 

% ----------------------------------------------------------
% Seção
% ----------------------------------------------------------
\section{Verificação e Validação}

Definição lorem ipsum dolor sit amet, consectetur adipisicing elit, sed do eiusmod
tempor incididunt ut labore et dolore magna aliqua. Ut enim ad minim veniam,
quis nostrud exercitation ullamco laboris nisi ut aliquip ex ea commodo
consequat. Duis aute irure dolor in reprehenderit in voluptate velit esse
cillum dolore eu fugiat nulla pariatur. Excepteur sint occaecat cupidatat non
proident, sunt in culpa qui officia deserunt mollit anim id est laborum.


\subsection{Análise estática}

Definição lorem ipsum dolor sit amet, consectetur adipisicing elit, sed do eiusmod
tempor incididunt ut labore et dolore magna aliqua. Ut enim ad minim veniam,
quis nostrud exercitation ullamco laboris nisi ut aliquip ex ea commodo
consequat. Duis aute irure dolor in reprehenderit in voluptate velit esse
cillum dolore eu fugiat nulla pariatur. Excepteur sint occaecat cupidatat non
proident, sunt in culpa qui officia deserunt mollit anim id est laborum.


\subsection{Inspeções}


\subsection{Revisões}


\subsection{Caso de teste}

% ----------------------------------------------------------
% Seção
% ----------------------------------------------------------
\section{Tipos de Problemas}


\subsection{Falha (\emph{Failure / problem})}


\subsection{Defeito (\emph{Fault / defect / bug})}


\subsection{Erro (\emph{Error / mistake})}

% ----------------------------------------------------------
% Seção
% ----------------------------------------------------------
\section{Categorias}


\subsection{Estrutural}


\subsection{Funcional}

% ----------------------------------------------------------
% Seção
% ----------------------------------------------------------
\section{Técnicas}


\subsection{White-box}


\subsection{Black-box}


\subsection{Grey-box}

% ----------------------------------------------------------
% Seção
% ----------------------------------------------------------
\section{Níveis}


\subsection{Unitário / componente}


\subsection{Integração / operação do sistema / funcional}


\subsection{Validação}


\subsection{Sistema}

% ----------------------------------------------------------
% Seção
% ----------------------------------------------------------
\section{Foco do teste}


\subsection{Conformidade}


\subsection{Configuração}


\subsection{Recuperação}


\subsection{Regressão}


\subsection{Estresse}


\subsection{Desempenho}


\subsection{Usabilidade}


\subsection{Acessibilidade}


\subsection{Internacionalização}

% ----------------------------------------------------------
% Seção
% ----------------------------------------------------------
\section{Ferramentas}

% ----------------------------------------------------------
% Seção
% ----------------------------------------------------------
\section{Fluxo de Teste em Diferentes Metodologias}

% ----------------------------------------------------------
% Capitulo
% ----------------------------------------------------------
\chapter{Inspeção}

% ----------------------------------------------------------
% Seção
% ----------------------------------------------------------
\section{Definição e Importância}

% ----------------------------------------------------------
% Seção
% ----------------------------------------------------------
\section{Processo de Inspeção (método Fagan)}

% ----------------------------------------------------------
% Seção
% ----------------------------------------------------------
\section{Funções de Inspeção}


% ----------------------------------------------------------
% Resultados
% ----------------------------------------------------------
\part{Exemplo (caso de uso?)}

% ---
% primeiro capitulo de Resultados
% ---
\chapter{Lectus lobortis condimentum}

% ---
\section{Vestibulum ante ipsum primis in faucibus orci luctus et ultrices
posuere cubilia Curae}
% ---

\lipsum[21-22]

% ---
% segundo capitulo de Resultados
% ---
\chapter{Nam sed tellus sit amet lectus urna ullamcorper tristique interdum
elementum}

\section{Pellentesque sit amet pede ac sem eleifend consectetuer}

\lipsum[24]

% ---
% Finaliza a parte no bookmark do PDF, para que se inicie o bookmark na raiz
% ---
\bookmarksetup{startatroot}% 
% ---

% ---
% Conclusão
% ---
\chapter*[Conclusão]{Conclusão}
\addcontentsline{toc}{chapter}{Conclusão}

\lipsum[31-33]

% ----------------------------------------------------------
% ELEMENTOS PÓS-TEXTUAIS
% ----------------------------------------------------------
\postextual


% ----------------------------------------------------------
% Referências bibliográficas
% ----------------------------------------------------------
%\bibliography{abntex2-modelo-references}
\bibliography{andrei-bibtex}

% ----------------------------------------------------------
% Glossário
% ----------------------------------------------------------
%
% Consulte o manual da classe abntex2 para orientações sobre o glossário.
%
%\glossary

% ----------------------------------------------------------
% Apêndices
% ----------------------------------------------------------

% ---
% Inicia os apêndices
% ---
\begin{apendicesenv}

% Imprime uma página indicando o início dos apêndices
\partapendices

% ----------------------------------------------------------
\chapter{Quisque libero justo}
% ----------------------------------------------------------

\lipsum[50]

% ----------------------------------------------------------
\chapter{Nullam elementum urna vel imperdiet sodales elit ipsum pharetra ligula
ac pretium ante justo a nulla curabitur tristique arcu eu metus}
% ----------------------------------------------------------
\lipsum[55-57]

\end{apendicesenv}
% ---


% ----------------------------------------------------------
% Anexos
% ----------------------------------------------------------

% ---
% Inicia os anexos
% ---
\begin{anexosenv}

% Imprime uma página indicando o início dos anexos
\partanexos

% ---
\chapter{Morbi ultrices rutrum lorem.}
% ---
\lipsum[30]

% ---
\chapter{Bla}
% ---

\lipsum[31]

% ---
\chapter{Fusce facilisis lacinia dui}
% ---

\lipsum[32]

\end{anexosenv}

%---------------------------------------------------------------------
% INDICE REMISSIVO
%---------------------------------------------------------------------

\printindex

\end{document}
